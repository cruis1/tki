From: pirmin.meier@company.ch\\
To: peter.hasler@company.ch

Hallo lieber Peter,\\\\
hiermit sende ich, wie angekündigt, die von Dir so dringend benötigten Informationen. 
Bezüglich der Weiterentwicklung im Bereich der Abteilung Forschung und Entwicklung kann ich Dir folgendes mitteilen. Ich habe mit dem Bereichsleiter T\&E unserer Division gestern ein intensives Gespräch darüber geführt. Mit Sicherheit konnte er mir auch noch nicht viel bestätigen, was aber schon feststeht und sich sicher nicht mehr ändern wird ist folgendes: Die jetzigen Forschung und Entwicklung Räumlichkeiten werden aufgegeben und ein neues, grösseres Labor am bestehenden Firmengebäude angebaut. Dieser Anbau dürfen wir aber frühestens 2022 erwarten. Die jetzigen zur Verfügung stehenden Messmittel werden zum grossen Teil ersetzt werden, wir werden neue Digigtal Signal Analyzer und Kathodenstrahloszilloskope mit sechs Kanälen erhalten. Die jetzigen werden, beginnend Ende 2017 etappenweise ersetzt. Genauere Informationen dazu später. Bezügich dem personellen Ausbau von unserem Team ist momentan eine Personalsteigerung zwischen 15-25\% in Diskussion. Diese Personen werden, beginnend 2018 rekrutiert und in die Abteilung eingegliedert.
Um ncoh Deine Frage wegen des zu verwendenden Transistors für die Emitterstufe (Schaltung T0455) zu beantworten, ich denke ein off-the-shelf BC547 wird dazu locker reichen. \\
Was ich nun von Dir noch benötige sind folgende Informationen:
Wie gross ist der Feedback-Widerstand des Op-Amp der Schaltung T0454?
Wann genau wirst Du im Herbst deinen WK leisten, damit ich die Personalplanung anpassen kann?
Wie ist der genaue Arbeitsstand im Gerdo-Projekt?\\\\
Vielen Dank für Deine Antwort\\\\
Gruss Pirmin

Pirmin Meier\\
El. Ing HTL\\
Abteilungsleiter F \& E\\
The Company AG\\
6300 Zug\\
Switzerland
   