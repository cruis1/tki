\newpage
\section*{Messbericht Schaltungsteil T0453} Verstärkerschaltung T0453 20.04.17

Der Schaltung vom Schaltungsteil T0453 wurde von mir (Pavel Datsyuk) am 20.04.17 mit Umgebungsbedingungen normal (22 Grade Celsius, 60 Prozentes relativer Luftfeuchtigkeiten) augemessen. Der Resultat der Messung ware wie erwartet positiv ausgefallen. Ich haben die Messunge exakt gleich gemachten auch noch einmal bei kühlen Temperatur im Kühler bei 10 Graden Celsius und 70 Prozenten Luftfeuchtigkeiten und noch bei 5 und 0 und -5 Graden Celsius und bei alle 75 Prozente relative Luftfeuchtigkeiten. Auch habe ich bei wärmere Bedingungens gemacht mit den Wärmeschrank. Dabei waren die Temperatur 30 Graden Celsius, 40 Graden, 50 Graden und 60 Graden und auch 70 Graden Celsius bei immer geichen Luftfeuchtigkeiten von 55 Prozente relativen. Die Messsergenbnissen finden wir in untener Tabelle aufgeführt. Anzumerke ist, dass ich für den Messungen das FLUKE 233, den Kathodenstrahloszilloskop Tektronix MDO3104 und den Signalgenerator SG554 von Horatio Enterprises verwenden habe.

Messergebnissen:

\begin{center}
  \begin{tabular}{ | p{2.7cm} | p{3.4cm} | p{1.2cm} | p{1.2cm} | p{1.2cm} | p{1.2cm} |}
   \hline
   \textbf{Temperatur / \textdegree C} & \textbf{rel. Luftfeuchte in \%} & \textbf{v01/ V} & \textbf{v02/ V} & \textbf{v03/ V} & \textbf{v04/ V}\\ \hline
   22 & 60 & 5.01 & 1.75 & 3.31 & 2.54\\ \hline
   -5 & 75 & 4.89 & 1.74 & 3.32 & 2.54\\ \hline
   00 & 57 & 4.92 & 1.76 & 3.31 & 2.54\\ \hline
   05 & 75 & 4.95 & 1.75 & 3.31 & 2.54\\ \hline
   30 & 55 & 5.05 & 1.73 & 3.31 & 2.55\\ \hline
   40 & 55 & 5.13 & 1.76 & 3.33 & 2.55\\ \hline
   50 & 55 & 5.25 & 1.74 & 3.35 & 2.56\\ \hline
   60 & 35 & 5.34 & 1.74 & 3.36 & 2.57\\ \hline
   70 & 55 & 5.39 & 1.74 & 3.37 & 2.58\\ \hline
  \end{tabular}
\end{center}

Auswertung:
Alle relevanten Schaltungsparameter sind genüg temperaturstabilen, so dass der Schaltungsteilen T0453 in der aktuellen Form so eingesetzt werd kann. Aufgrund den Spannungsdrift bei v01 muss aber in derdirekt angehängten Operationsverstärkerschaltung T0454 der Widerstand .... angepasst werden.



Pavel Datsyuk

\newpage