\chapter{Schlussfolgerungen und Empfehlungen}
Die Schlussfolgerungen und Empfehlungen werden anhand einer Liste aufgeführt. Sie beziehen sich nicht nur auf das Assessment-Projekt selber, sondern vielmehr auf alle Inhalte aud Ausfgaben des Moduls Teamkommunikation für Ingenieure. 

\begin{itemize}
\item Dem Kennenlernprozess am Anfang einer Teamarbeit ist grösseren Wert beizumessen. So können früh Stärken und Schwächen der Teammitglieder erkannt werden und die Rollenverteilung entsprechend vorgenommen werden. Dies beinhaltet auch, erledigte Aufgaben anfänglich gemeinsam zu begutachten, so dass sich jedes Teammitglied ein Bild der Leistung der anderen machen kann. Denn es reicht oft nicht, einfach auf die Aussagen der Teammitglieder betreffend ihres Könnens zu vertrauen.
\item Das Team kann sich das Leben extrem erleichtern, wenn die Aufgaben exakt nach den Stärken der Mitglieder verteilt werden. So ist das Lernpotenzial eines jeden zwar kleiner, der gesamte Aufwand aber auch erheblich geringer.
\item Wird ein Koordinator oder Teamchef bestimmt, sollte sich dieser nicht zu schade sein, seine Führungsaufgabe auch aufrichtig wahrzunehmen. Auch mit einer gewissen angewandten Strenge kann er immer noch einen freundschaftlichen Umgang mit den anderen Teammitgliedern pflegen und wird von diesen nicht als schlechte Person wahrgenommen.
\item Bei zukünftigen Arbeiten muss unbedingt früher mit Dokumentieren, also dem Schreiben des Projektberichtes begonnen werden. Die Arbeit muss sowieso erledigt werden, ob dies laufend oder ganz auf den letzten Drücker getan wird. Der grosse Nachteil wenn es erst am Schluss erledigt wird, ist halt, dass in der verstrichenen Zeit sehr viel Information verloren geht und mühsam aufgearbeitet werden muss.
\item Wenn schon eine detaillierte Planung inklusive Meilensteinen und Beendigungszeiten erstellt wird, sollte das Team auch versucht sein, sich an diese zu halten. Es ist nicht die Aufgabe des Koordinators, jede Woche aufs Neue den Anderen mitteilen zu müssen was im Moment zu tun ist und dass sie doch bitte entsprechend der Planung weiterarbeiten sollen, besonders wenn der Koordinator aufgrund externer Gründe nicht anwesend sein kann.
\item Dank einem toleranten Dozenten konnte eine Abgabefristerstreckung von einer Woche erreicht werden, ohne diese wäre der Projektbericht in einem miserablen Zustand gewesen und eine genügende Note äusserst unwahrscheinlich gewesen.
\end{itemize}

\section{Aufwand und Kosten}
Im Nachhinein kann gesagt werden, dass der zeitliche Aufwand eher geringer ausfiel als anfänglich eingeplant. So haben die Mitglieder des Teams in etwa 2/3 der eingeplanten Zeit aufgewendet. Hätte man bei jedem Arbeitsschritt mehr Wert auf die Qualität gelegt, wären die fürs Projekt eingesetzten Stunden noch näher and die Geplanten gekommen. Mit den Kosten verhält es sich entsprechend proportional. 

