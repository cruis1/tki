\chapter{Reflexion}

\section{Lesson learned}

Da dieses Projekt im Rahmen des Teamkommunikation Moduls durchgeführt wurde, ist auf diesen Bereich ein besonderes Augenmerk zu richten. Nach den drei ausführlichen Teamreviews und Lernbilanzen war es nötig, sich stark mit dem Team und den Arbeitsprozessen auseinanderzusetzen. 

\subsubsection{Technologieschwierigkeiten} \label{sec:techn}

Dadurch, dass wir uns am Anfang dafür entschieden haben, alles mit Latex und Git zu machen, wurde mehr Zeit verwendet den Umgang mit diesen Technologien in den Griff zu bekommen, als für die tatsächlich geforderten Aufgaben. Obwohl die beiden Tools für mindestens zwei Teammitgliedern ein Ansporn waren, kam es soweit, dass wir teilweise dermassen viel zu tun hatten, dass die neuen Tools zu einer grossen Last wurden. Jedoch klappte es jede Woche etwas besser und im Nachhinein kann mit gutem Gewissen gesagt werden, dass die beiden Tool sehr gut beherrscht wurden. Die Technologien sind nicht mehr eine Last, sondern eine Bereicherung und tragen momentan zur Effizienzsteigerung bei.

\subsubsection {Organisationsprobleme} \label{sec:orgprob}
Durch die im Abschnitt (\ref{sec:techn}) beschriebenen Schwierigkeiten kam es dazu, dass wir Anfangs oft alles zusammen gemacht haben. Jeder einzelne Auftrag wurde zusammen sitzend erledigt. Dieses Vorgehen war natürlich sehr ineffizient. Im Teamvertrag legten wir eine Aufgabenteilung fest, jedoch war diese völlig unrealistisch, da sich natürliche Teamrollen aufgrund verschiedener persönlicher stärken/schwächen entwickelten. Als Beispiel ist zu nennen, dass wir anfangs Gerome den Umgang mit MS-Projekt anvertrauten und im laufe der Zeit sich herausstellte, dass Pascal sehr gut in der Organisations- und Führungstätigkeit war. Es machte also sinn, dass Pascal den MS-Project Teil vollständig übernahm. Auch schränkten Grammatikschwierigkeiten und das technische Know-how eines Teammitgliedes die Möglichkeiten der zu erledigenden Aufgaben stark ein. \\
Wir hatten Glück, dass sich Pascal sehr in Zeug legte und einen grossen Teil der Aufgaben auf sich nahm. Viele latente Konflikte waren während dieser Zeit sehr wahrscheinlich ein nicht zu vernachlässigender Teil des Teams. Nichtsdestotrotz hatten wir keine nennenswerten offenen Konflikte zu klären. Das keine offenen Konflikte aufgetreten sind liegt daran, dass kein Teammitglied es darauf ankommen liess. Die Gründe dafür mögen bei jedem unterschiedlich gewesen sein und teilweise auch Einzug in die Lernbilanzen/Teamreviews gefunden haben. Aus diesem Grunde wird darauf nicht weiter eingegangen. Der Lerneffekt ist dementsprechend bei jedem Teammitglied ein anderer.

\section{Verbesserungspotenzial}

Im nachhinein wäre wichtig gewesen, dass wir bei der Entscheidung, dass Latex und Git zum Einsatz kommen auch sehr strikt darauf achten, dass jeder im Selbststudium das lernt, was nötig ist, um effizient mitarbeiten zu können. \\
Trotzdem gab es Schwierigkeiten im Team, die grundsätzlicher Natur waren und nicht einfach geändert werden konnte (Siehe \ref{sec:orgprob}). Viele stärken und schwächen mussten akzeptiert und falls möglich möglichst effizient eingesetzt werden. 