\chapter{Bewertung Probanden}
\section{Einleitung}
Um die Tauglichkeit unserer ausgedachten Übungen, wie auch deren Schwierigkeitsgrad einschätzen zu können, wurde das Verhalten der Bewerber während dem  Assessment von uns beobachtet und mit den im Kapitel \ref{Assessment} auf Seite \pageref{Assessment} beschriebenen Bewertungskriterien eingeschätzt.\\
Die Bewertung wird unten, nach den geprüften Personen sortiert, aufgeführt.

\section{Tomo Bogdanovic}
\subsection{Resultate Übung 1: Analytisches und systematisches Denken, Lernbereitschaft und Lernfähigkeit}


\subsubsection{Verschafft er sich zuerst einen Überblick?}
\begin{tabular}{| l | c | c | c |}
  \hline	
  \textbf{Ausprägung} & \textbf{Punkte lb \& lf} & \textbf{Punkte lg \& an} & \textbf{Punkte sm \& so} \\
  \hline  		
  Schafft sich keinen Überblick & 0 & 0 & 0 \\ 
  \hline
  Schafft sich fast keinen Überblick & 1 & 0 & 1 \\ 
  \hline
  Schafft sich Überblick & \circletext{2} & \circletext{0} & \circletext{2} \\
  \hline  
  Verschafft sich sehr genauen Überblick & 3 & 0 &  3 \\
  \hline  
\end{tabular}

\subsubsection{Ist eine Methodik zu erkennen oder wird ziellos gesucht?}
\begin{tabular}{| l | c | c | c |}
  \hline	
  \textbf{Ausprägung} & \textbf{Punkte lb \& lf} & \textbf{Punkte lg \& an} & \textbf{Punkte sm \& so} \\
  \hline  		
  Gar keine Methodik & 0  & 0 & 0 \\ 
  \hline
  Nur wenig Methodik & 1 & 0 & 1 \\ 
  \hline
  Gute Methodik & \circletext{2} & \circletext{0} & \circletext{2} \\
  \hline  
  Sehr strukturierte Vorgehensweise & 3 & 0 &  3 \\
  \hline  
\end{tabular}

\subsubsection{Wie sicher fühlt sich die Person während der Präsentation?}
\begin{tabular}{| l | c | c | c |}
  \hline	
  \textbf{Ausprägung} & \textbf{Punkte lb \& lf} & \textbf{Punkte lg \& an} & \textbf{Punkte sm \& so} \\
   \hline  		
  Sehr unsicher & 0  & 0 & 0 \\ 
  \hline
  Eher unsicher & 0 & 0 & 0 \\ 
  \hline
  Eher sicher & \circletext{0} & \circletext{1} & \circletext{1} \\
  \hline  
  Sehr selbstsicher & 0 & 2 &  2 \\
  \hline  
\end{tabular}


\subsubsection{Wie genau/verständlich konnte der Prozess erklärt werden. Wurden die wichtigsten Punkte erwähnt?}
\begin{tabular}{| l | c | c | c |}
  \hline	
  \textbf{Ausprägung} & \textbf{Punkte lb \& lf} & \textbf{Punkte lg \& an} & \textbf{Punkte sm \& so} \\
  \hline  		
  Hat Prozess nicht verstanden & 0 & 0 & 0 \\ 
  \hline
  Hat Prozess etwas verstanden & 1 & 1 & 1 \\ 
  \hline
  Hat Prozess gut verstranden & \circletext{2} & \circletext{2} & \circletext{2} \\
  \hline  
  Hat Prozess hervorragend verstanden & 3 & 3 &  3 \\
  \hline  
\end{tabular}

\subsubsection{Wurden Fremdwörter verwendet oder alles in eigene Worte übersetzt?}
\begin{tabular}{| l | c | c | c |}
  \hline	
  \textbf{Ausprägung} & \textbf{Punkte lb \& lf} & \textbf{Punkte lg \& an} & \textbf{Punkte sm \& so} \\
  \hline  		
  Keine Fremdwörter verwendet & 0 & 0 & 0 \\ 
  \hline
  Wenig Fremdwörter verwendet & 1 & 0 & 0 \\ 
  \hline
  Einige Fremdwörter verwendet & \circletext{2} & \circletext{0} & \circletext{0} \\
  \hline  
  Sehe viele Fremdwörter verwendet & 3 & 0 &  0 \\
  \hline  
\end{tabular}

\subsubsection{Wurde visuell gearbeitet? (Damit sind die Notizen inbegriffen)}
\begin{tabular}{| l | c | c | c |}
  \hline	
  \textbf{Ausprägung} & \textbf{Punkte lb \& lf} & \textbf{Punkte lg \& an} & \textbf{Punkte sm \& so} \\
  \hline  		
  Keine & 0  & 0 & 0 \\ 
  \hline
  Kaum & 1 & 1 & 1 \\ 
  \hline
  Einige & \circletext{2} & \circletext{2} & \circletext{2} \\
  \hline  
  Sehe viele & 3 & 3 & 3 \\
  \hline  
\end{tabular}

\subsubsection{Wurden die Erklärungen strukturiert?}
\begin{tabular}{| l | c | c | c |}
  \hline	
  \textbf{Ausprägung} & \textbf{Punkte lb \& lf} & \textbf{Punkte lg \& an} & \textbf{Punkte sm \& so} \\
  \hline  		
  Gar nich & 0 & 0 & 0 \\ 
  \hline
  Kaum & 1 & 1 & 1 \\ 
  \hline
  Genügend & 2 & 2 & 2 \\
  \hline  
  Sehr gute Struktur & \circletext{3} & \circletext{3} & \circletext{3} \\
  \hline  
\end{tabular}

\subsubsection{Wie viel wurde über den Prozess gesprochen? Wurde unnötig über Nebensächlichkeiten geredet?}
\begin{tabular}{| l | c | c | c |}
  \hline	
  \textbf{Ausprägung} & \textbf{Punkte lb \& lf} & \textbf{Punkte lg \& an} & \textbf{Punkte sm \& so} \\
  \hline  		
  Viele Nebensächlichkeiten & 0  & 0 & 0 \\ 
  \hline
  Einige Nebensächlichkeiten & 1 & 1 & 1 \\ 
  \hline
  Kaum Nebensächlichkeiten & 2 & 2 & 2 \\
  \hline  
  Keine Nebensächlichkeiten & \circletext{3} & \circletext{3} & \circletext{3} \\
  \hline  
\end{tabular}

\subsubsection{Wirkte die Person interessiert?}
\begin{tabular}{| l | c | c | c |}
  \hline	
  \textbf{Ausprägung} & \textbf{Punkte lb \& lf} & \textbf{Punkte lg \& an} & \textbf{Punkte sm \& so} \\
  \hline  		
  Gar nicht & 0  & 0 & 0 \\ 
  \hline
  Etwas & 1 & 0 & 0 \\ 
  \hline
  Interessiert & 2 & 0 & 0 \\
  \hline  
  Sehr Interessiert & \circletext{3} & \circletext{0} & \circletext{0} \\
  \hline  
\end{tabular}


\subsubsection{Summe der erreichten Punkte}
\begin{center}
  \begin{tabular}{ | p{7cm} | p{3cm} | p{3cm} |}
   \hline
   \textbf{Kernkompetenz} & \textbf{erreichte Punkte} & \textbf{maximale Punkte} \\ \hline
   Analytisches und systematisches Denken & 11 & 14\\ \hline
  Lernbereitschaft und Lernfähigkeit & 19 & 24\\ \hline
   Selbstmanagement und Selbstorganisation & 15 & 20\\ \hline
   \textbf{Total Übung 1} & \textbf{45} & \textbf{58}\\ \hline
  \end{tabular}
\end{center}

\subsubsection{Vorgehen des Kandidaten} \label{bemtom}
Anfangs schaute sich Tomo den Wikipedia Artikel an und las die Einleitung und den Theorieteil durch.  Anschliessend suchte er im Internet einfachere Erklärungen. Erst nach 5 Minuten einlesen begann er sich Notizen zu machen. Gegen den Schluss suchte er noch Bilder vom Prozess selbst und las noch etwas zur Geschichte der Polymerase Kettenreaktion.
\newpage
\subsubsection{Anmerkungen /Präsentation} \label{bemtom}

Tomo hatte vorher kein Vorwissen. Er meinte, dass er sich an eine Animation erinnern kann, wo DNA repliziert wird. Ich gehe davon aus, dass die DNA-Polymerase während der Replikation der DNA gezeigt wurde. Trotz des Videos konnte er nicht mit Sicherheit sagen, wie die DNA-Repliziert wird. Seine Vermutung war anfangs, dass die Primer aneinander gehängt werden und so den ganzen Strang replizieren. Er kam erst mit Nachfragen darauf, dass dies wahrscheinlich nicht der Fall sein wird. \\
Sonst wurde der Prozess gut erklärt. Die Präsentation war gut strukturiert und er behalf sich mit visuellen Darstellungen.
\vspace{0.5cm}

%\newpage
\subsection{Übung 2: Selbstmanagement und Selbstorganisation}
\subsubsection{Resultate der Bewertungskriterien}
\begin{center}
  \begin{longtable}{ | p{11cm} | p{2cm} |}
   \hline
   \textbf{macht sich der Proband Notizen?} & \textbf{Punkte} \\ \hline
   macht keine Notizen & 0 \\ \hline
   macht wenige Notizen & 1 \\ \hline
   macht viele Notizen & \circletext{2} \\ \hline
   notiert sich alles  & 3\\ \hline
   \textbf{schafft sich der Proband Übersicht?} & \textbf{Punkte} \\ \hline
   schafft sich keine Übersicht & 0 \\ \hline
   schafft sich fast keine Übersicht & 1 \\ \hline
   schafft sich Übersicht & \circletext{2} \\ \hline
   studiert die Aufgabenstellungen gründlich  & 3\\ \hline
   \textbf{erstellt der Proband eine Zeitplanung?} & \textbf{Punkte} \\ \hline
   erstellt keine Zeitplanung & 0 \\ \hline
   erstellt so etwas wie eine Zeitplanung & \circletext{1} \\ \hline
   erstellt eine gute Zeitplanung & 2 \\ \hline
   erstellt detail. Zeitplanung inkl. Reservezeiten  & 3\\ \hline
   \textbf{wendet der Proband das Eisenhower-Prinzip an?} & \textbf{Punkte} \\ \hline
   wendet Eisenhower-Prinzip nicht an & 0 \\ \hline
   versucht das Eisenhower-Prinzip anzuwenden & \circletext{1} \\ \hline
   wendet Eisenhower-Prinzip richtig an & 2 \\ \hline
   wendet Eisenhower-Prinzip korrekt an und leitet dadurch Konsequenzen für die Bearbeitung der Aufgaben ab  & 3\\ \hline
   \textbf{notiert sich der Proband Fragen?} & \textbf{Punkte} \\ \hline
   notiert keine Fragen & \circletext{0} \\ \hline
   notiert sich wenige relevante Fragen & 1 \\ \hline
   notiert einige relevante Fragen & 2 \\ \hline
   notiert und stellt alle relevanten Fragen  & 3\\ \hline
   \textbf{informiert der Proband über seinen Arbeitsstand?} & \textbf{Punkte} \\ \hline
   informiert den Prüfer nicht über den Arbeitsstand & \circletext{0} \\ \hline
   gibt fast keine Informationen weiter & 1 \\ \hline
   gibt viele Informationen an den Prüfer weiter & 2 \\ \hline
   gibt alle relevanten Informationen zeitgerecht an den Prüfer weiter & 3\\ \hline
%  \end{longtable}
%\end{center}
%  \begin{center}
%
%  \begin{tabular}{ | p{11cm} | p{2cm} |}
  % \hline
   \textbf{Ausprägung analytisches Denken} & \textbf{Punkte} \\ \hline
   es fällt dem Geprüften sehr schwer die Aufgabenstellungen zu verstehen & 0 \\ \hline
   der Geprüfte bei einigen Aufgabenstellungen Mühe sie zu verstehen  & 1 \\ \hline
   der Geprüfte hat wenig Mühe die Aufgabenstellungen zu verstehen & \circletext{2} \\ \hline
   dem Geprüften waren alle Aufgabenstellungen sofort klar & 3\\ \hline
   \textbf{Ausprägung systematisches Denken} & \textbf{Punkte} \\ \hline
   der Geprüfte hat grosse Mühe, die Verknüpfungen zwischen den Aufgaben zu verstehen & 0 \\ \hline
    der Geprüfte hat teilweise Mühe, die Verknüpfungen zwischen den Aufgaben zu verstehen & 1 \\ \hline
   der Geprüfte findet und versteht die Abhängigkeiten der Aufgaben & \circletext{2} \\ \hline
   findet und versteht die Abhängigkeiten der Aufgaben auf Anhieb & 3\\ \hline
  \end{longtable}
\end{center}
\vspace{0.1cm}
\subsubsection{Summe der erreichten Punkte}

\begin{center}
  \begin{tabular}{ | p{7cm} | p{3cm} | p{3cm} |}
   \hline
   \textbf{Kernkompetenz} & \textbf{erreichte Punkte} & \textbf{maximale Punkte} \\ \hline
   Selbstmanagement und Selbstorganisation & 6 & 18\\ \hline
   Analytisches und systematisches Denken & 4 & 6\\ \hline
   \textbf{Total Übung 2} & \textbf{10} & \textbf{24}\\ \hline
  \end{tabular}
\end{center}

\subsubsection{Anmerkungen}
Obwohl dem Probanden deutlich mitgeteilt wurde, dass Fragen erlaubt sind und auch einige Unklarheiten herrschten, vergass er, fragen zu stellen. Als ich ihm im Feedback-Gespräch mitteilte, dass die Aufgaben gar nicht ohne Nachfragen beim Prüfer zu lösen seien, zeigte er sich beschämt, dass er keine Fragen gestellt hatte.



\section{Michel André Nyffenegger}
\subsection{Resultate Übung 1: Analytisches und systematisches Denken, Lernbereitschaft und Lernfähigkeit}

\subsubsection{Verschafft er sich zuerst einen Überblick?}
\begin{tabular}{| l | c | c | c |}
  \hline	
  \textbf{Ausprägung} & \textbf{Punkte lb \& lf} & \textbf{Punkte lg \& an} & \textbf{Punkte sm \& so} \\
  \hline  		
  Schafft sich keinen Überblick & \circletext{0} & \circletext{0} & \circletext{0} \\ 
  \hline
  Schafft sich fast keinen Überblick & 1 & 0 & 1 \\ 
  \hline
  Schafft sich Überblick & 2 & 0 & 2 \\
  \hline  
  Verschafft sich sehr genauen Überblick & 3 & 0 &  3 \\
  \hline  
\end{tabular}

\subsubsection{Ist eine Methodik zu erkennen oder wird ziellos gesucht?}
\begin{tabular}{| l | c | c | c |}
  \hline	
  \textbf{Ausprägung} & \textbf{Punkte lb \& lf} & \textbf{Punkte lg \& an} & \textbf{Punkte sm \& so} \\
  \hline  		
  Gar keine Methodik & 0 & 0 & 0 \\ 
  \hline
  Nur wenig Methodik & \circletext{1} & \circletext{0} & \circletext{1} \\ 
  \hline
  Gute Methodik & 2 & 0 & 2 \\
  \hline  
  Sehr strukturierte Vorgehensweise & 3 & 0 &  3 \\
  \hline  
\end{tabular}

\subsubsection{Wie sicher fühlt sich die Person während der Präsentation?}
\begin{tabular}{| l | c | c | c |}
  \hline	
  \textbf{Ausprägung} & \textbf{Punkte lb \& lf} & \textbf{Punkte lg \& an} & \textbf{Punkte sm \& so} \\
  \hline  		
  Sehr unsicher & 0  & 0 & 0 \\ 
  \hline
  Eher unsicher & 0 & 0 & 0 \\ 
  \hline
  Eher sicher & \circletext{0} & \circletext{1} & \circletext{1} \\
  \hline  
  Sehr selbstsicher & 0 & 2 &  2 \\
  \hline  
\end{tabular}

\subsubsection{Wie genau/verständlich konnte der Prozess erklärt werden. Wurden die wichtigsten Punkte erwähnt?}
\begin{tabular}{| l | c | c | c |}
  \hline	
  \textbf{Ausprägung} & \textbf{Punkte lb \& lf} & \textbf{Punkte lg \& an} & \textbf{Punkte sm \& so} \\
  \hline  		
  Hat Prozess nicht verstanden & \circletext{0} & \circletext{0} & \circletext{0} \\ 
  \hline
  Hat Prozess etwas verstanden & 1 & 1 & 1 \\ 
  \hline
  Hat Prozess gut verstranden & 2 & 2 & 2 \\
  \hline  
  Hat Prozess hervorragend verstanden & 3 & 3 &  3 \\
  \hline  
\end{tabular}

\subsubsection{Wurden Fremdwörter verwendet oder alles in eigene Worte übersetzt?}
\begin{tabular}{| l | c | c | c |}
  \hline	
  \textbf{Ausprägung} & \textbf{Punkte lb \& lf} & \textbf{Punkte lg \& an} & \textbf{Punkte sm \& so} \\
  \hline  		
  Keine Fremdwörter verwendet & 0  & 0 & 0 \\ 
  \hline
  Wenig Fremdwörter verwendet & 1 & 0 & 0 \\ 
  \hline
  Einige Fremdwörter verwendet & 2 & 0 & 0 \\
  \hline  
  Sehe viele Fremdwörter verwendet & \circletext{3} & \circletext{0} &  \circletext{0} \\
  \hline  
\end{tabular}

\subsubsection{Wurde visuell gearbeitet? (Damit sind die Notizen inbegriffen)}
\begin{tabular}{| l | c | c | c |}
  \hline	
  \textbf{Ausprägung} & \textbf{Punkte lb \& lf} & \textbf{Punkte lg \& an} & \textbf{Punkte sm \& so} \\
  \hline  		
  Keine & 0  & 0 & 0 \\ 
  \hline
  Kaum & 1 & 1 & 1 \\ 
  \hline
  Einige & 2 & 2 & 2 \\
  \hline  
  Sehe viele & \circletext{3} & \circletext{3} & \circletext{3} \\
  \hline  
\end{tabular}

\subsubsection{Wurden die Erklärungen strukturiert?}
\begin{tabular}{| l | c | c | c |}
  \hline	
  \textbf{Ausprägung} & \textbf{Punkte lb \& lf} & \textbf{Punkte lg \& an} & \textbf{Punkte sm \& so} \\
  \hline  		
  Gar nich & 0  & 0 & 0 \\ 
  \hline
  Kaum & 1 & 1 & 1 \\ 
  \hline
  Genügend & 2 & 2 & 2 \\
  \hline  
  Sehr gute Struktur & \circletext{3} & \circletext{3} & \circletext{3} \\
  \hline  
\end{tabular}

\subsubsection{Wie viel wurde über den Prozess gesprochen? Wurde unnötig über Nebensächlichkeiten geredet?}
\begin{tabular}{| l | c | c | c |}
  \hline	
  \textbf{Ausprägung} & \textbf{Punkte lb \& lf} & \textbf{Punkte lg \& an} & \textbf{Punkte sm \& so} \\
  \hline  		
  Viele Nebensächlichkeiten & \circletext{0} & \circletext{0} & \circletext{0} \\ 
  \hline
  Einige Nebensächlichkeiten & 1 & 1 & 1 \\ 
  \hline
  Kaum Nebensächlichkeiten & 2 & 2 & 2 \\
  \hline  
  Keine Nebensächlichkeiten & 3 & 3 & 3 \\
  \hline  
\end{tabular}

\subsubsection{Wirkte die Person interessiert?}
\begin{tabular}{| l | c | c | c |}
  \hline	
  \textbf{Ausprägung} & \textbf{Punkte lb \& lf} & \textbf{Punkte lg \& an} & \textbf{Punkte sm \& so} \\
  \hline  		
  Gar nicht & 0 & 0 & 0 \\ 
  \hline
  Etwas & 1 & 0 & 0 \\ 
  \hline
  Interessiert & \circletext{2} & \circletext{0} & \circletext{0} \\
  \hline  
  Sehr Interessiert & 3 & 0 & 0 \\
  \hline  
\end{tabular}


\subsubsection{Summe der erreichten Punkte}
\vspace{-0.3cm}
\begin{center}
  \begin{tabular}{ | p{7cm} | p{3cm} | p{3cm} |}
   \hline
   \textbf{Kernkompetenz} & \textbf{erreichte Punkte} & \textbf{maximale Punkte} \\ \hline
   Analytisches und systematisches Denken & 7 & 14\\ \hline
  Lernbereitschaft und Lernfähigkeit & 12 & 24\\ \hline
   Selbstmanagement und Selbstorganisation & 8 & 20\\ \hline
   \textbf{Total Übung 1} & \textbf{27} & \textbf{58}\\ \hline
  \end{tabular}
\end{center}
\vspace{0.3cm}
\subsubsection{Vorgehen des Kandidaten} \label{bemmic}
Michel ging anfangs, wie alle anderen Testkandidaten, dazu über den Wikipedia Artikel zu lesen. Gleich am Anfang begann er damit sich ein Mindmap zu zeichnen. Er las dann die Einleitung des Wikipedia Artikels und schliesslich den Praxis-Teil. Damit verbrachte er die ersten 6 Minuten und scrollte auch nicht weiter nach unten um zu sehen, was noch wichtigeres stehen könnte. Somit sah er den Teil, wo der theoretische Ablauf erklärt wurde erst nach 6 Minuten. Als er diesen Teil dann auch gelesen hatte, scrollte er schlussendlich alles durch. Die Zeit war dann bereits abgelaufen und das Resultat war dann auch wie erwartet nicht zufriedenstellend.

\subsubsection{Präsenation}

Der Einstieg der Präsentation war zwar sehr gelungen, auch konnte er eine gutes Mindmap vorweisen und die technischen Begriffe erwähnen, jedoch wusste er nicht, was diese bedeuten und den Prozess hatte er nicht ansatzweise verstanden. 

\subsection{Übung 2: Selbstmanagement und Selbstorganisation}
\subsubsection{Resultate der Bewertungskriterien}
\begin{center}
  \begin{longtable}{ | p{11cm} | p{2cm} |}
   \hline
   \textbf{macht sich der Proband Notizen?} & \textbf{Punkte} \\ \hline
   macht keine Notizen & 0 \\ \hline
   macht wenige Notizen & \circletext{1} \\ \hline
   macht viele Notizen & 2 \\ \hline
   notiert sich alles  & 3\\ \hline
   \textbf{schafft sich der Proband Übersicht?} & \textbf{Punkte} \\ \hline
   schafft sich keine Übersicht & 0 \\ \hline
   schafft sich fast keine Übersicht & \circletext{1} \\ \hline
   schafft sich Übersicht & 2 \\ \hline
   studiert die Aufgabenstellungen gründlich  & 3\\ \hline
   \textbf{erstellt der Proband eine Zeitplanung?} & \textbf{Punkte} \\ \hline
   erstellt keine Zeitplanung & 0 \\ \hline
   erstellt so etwas wie eine Zeitplanung & \circletext{1} \\ \hline
   erstellt eine gute Zeitplanung & 2 \\ \hline
   erstellt detail. Zeitplanung inkl. Reservezeiten  & 3\\ \hline
   \textbf{wendet der Proband das Eisenhower-Prinzip an?} & \textbf{Punkte} \\ \hline
   wendet Eisenhower-Prinzip nicht an & 0 \\ \hline
   versucht das Eisenhower-Prinzip anzuwenden & 1 \\ \hline
   wendet Eisenhower-Prinzip richtig an & \circletext{2} \\ \hline
   wendet Eisenhower-Prinzip korrekt an und leitet dadurch Konsequenzen für die Bearbeitung der Aufgaben ab  & 3\\ \hline
   \textbf{notiert sich der Proband Fragen?} & \textbf{Punkte} \\ \hline
   notiert keine Fragen & \circletext{0} \\ \hline
   notiert sich wenige relevante Fragen & 1 \\ \hline
   notiert einige relevante Fragen & 2 \\ \hline
   notiert und stellt alle relevanten Fragen  & 3\\ \hline
   \textbf{informiert der Proband über seinen Arbeitsstand?} & \textbf{Punkte} \\ \hline
   informiert den Prüfer nicht über den Arbeitsstand & 0 \\ \hline
   gibt fast keine Informationen weiter & \circletext{1} \\ \hline
   gibt viele Informationen an den Prüfer weiter & 2 \\ \hline
   gibt alle relevanten Informationen zeitgerecht an den Prüfer weiter & 3\\ \hline
%  \end{tabular}
%\end{center}
%\begin{center}
%  \begin{tabular}{ | p{11cm} | p{1cm} |}
%   \hline
   \textbf{Ausprägung analytisches Denken} & \textbf{Punkte} \\ \hline
   es fällt dem Geprüften sehr schwer die Aufgabenstellungen zu verstehen & 0 \\ \hline
   der Geprüfte bei einigen Aufgabenstellungen Mühe sie zu verstehen  & 1 \\ \hline
   der Geprüfte hat wenig Mühe die Aufgabenstellungen zu verstehen & \circletext{2} \\ \hline
   dem Geprüften waren alle Aufgabenstellungen sofort klar & 3\\ \hline
   \textbf{Ausprägung systematisches Denken} & \textbf{Punkte} \\ \hline
   der Geprüfte hat grosse Mühe, die Verknüpfungen zwischen den Aufgaben zu verstehen & 0 \\ \hline
    der Geprüfte hat teilweise Mühe, die Verknüpfungen zwischen den Aufgaben zu verstehen & \circletext{1} \\ \hline
   der Geprüfte findet und versteht die Abhängigkeiten der Aufgaben & 2 \\ \hline
   findet und versteht die Abhängigkeiten der Aufgaben auf Anhieb & 3\\ \hline
  \end{longtable}
\end{center}
\vspace{0.2cm}
\subsubsection{Summe der erreichten Punkte}

\begin{center}
\vspace{0.2cm}
  \begin{tabular}{ | p{7cm} | p{3cm} | p{3cm} |}
   \hline
   \textbf{Kernkompetenz} & \textbf{erreichte Punkte} & \textbf{maximale Punkte} \\ \hline
   Selbstmanagement und Selbstorganisation & 6 & 18\\ \hline
   Analytisches und systematisches Denken & 3 & 6\\ \hline
   \textbf{Total Übung 2} & \textbf{9} & \textbf{24}\\ \hline
  \end{tabular}
\end{center}

\subsubsection{Anmerkungen}
Der Proband hat frühzeitig gemerkt, dass er für das vollständige Lösen der Emitterschaltung mehr Angaben braucht. Trotzdem hat er dazu nicht in eine Frage gestellt, sonst hätte der Prüfer ihm das Datenblatt des Transistors ausgehändigt, auf dem alle benötigten Angaben aufgeführt sind.

\section{Matthias Schneider}

\subsection{Resultate Übung 1:Analytisches und systematisches Denken, Lernbereitschaft und Lernfähigkeit}

\subsubsection{Verschafft er sich zuerst einen Überblick?}
\begin{tabular}{| l | c | c | c |}
  \hline	
  \textbf{Ausprägung} & \textbf{Punkte lb \& lf} & \textbf{Punkte lg \& an} & \textbf{Punkte sm \& so} \\
  \hline  		
  Schafft sich keinen Überblick & 0  & 0 & 0 \\ 
  \hline
  Schafft sich fast keinen Überblick & 1 & 0 & 1 \\ 
  \hline
  Schafft sich Überblick & \circletext{2} & \circletext{0} & \circletext{2} \\
  \hline  
  Verschafft sich sehr genauen Überblick & 3 & 0 &  3 \\
  \hline  
\end{tabular}

\subsubsection{Ist eine Methodik zu erkennen oder wird ziellos gesucht?}
\begin{tabular}{| l | c | c | c |}
  \hline	
  \textbf{Ausprägung} & \textbf{Punkte lb \& lf} & \textbf{Punkte lg \& an} & \textbf{Punkte sm \& so} \\
  \hline  		
  Gar keine Methodik & 0 & 0 & 0 \\ 
  \hline
  Nur wenig Methodik & 1 & 0 & 1 \\ 
  \hline
  Gute Methodik & 2 & 0 & 2 \\
  \hline  
  Sehr strukturierte Vorgehensweise & \circletext{3} & \circletext{0} &  \circletext{3} \\
  \hline  
\end{tabular}

\subsubsection{Wie sicher fühlt sich die Person während der Präsentation?}
\begin{tabular}{| l | c | c | c |}
  \hline	
  \textbf{Ausprägung} & \textbf{Punkte lb \& lf} & \textbf{Punkte lg \& an} & \textbf{Punkte sm \& so} \\
  \hline  		
  Sehr unsicher & 0  & 0 & 0 \\ 
  \hline
  Eher unsicher & 0 & 0 & 0 \\ 
  \hline
  Eher sicher & 0 & 1 & 1 \\
  \hline  
  Sehr selbstsicher & \circletext{0} & \circletext{2} &  \circletext{2} \\
  \hline  
\end{tabular}

\subsubsection{Wie genau/verständlich konnte der Prozess erklärt werden. Wurden die wichtigsten Punkte erwähnt?}
\begin{tabular}{| l | c | c | c |}
  \hline	
  \textbf{Ausprägung} & \textbf{Punkte lb \& lf} & \textbf{Punkte lg \& an} & \textbf{Punkte sm \& so} \\
  \hline  		
  Hat Prozess nicht verstanden & 0  & 0 & 0 \\ 
  \hline
  Hat Prozess etwas verstanden & 1 & 1 & 1 \\ 
  \hline
  Hat Prozess gut verstranden & 2 & 2 & 2 \\
  \hline  
  Hat Prozess hervorragend verstanden & \circletext{3} & \circletext{3} &  \circletext{3} \\
  \hline  
\end{tabular}

\subsubsection{Wurden Fremdwörter verwendet oder alles in eigene Worte übersetzt?}
\begin{tabular}{| l | c | c | c |}
  \hline	
  \textbf{Ausprägung} & \textbf{Punkte lb \& lf} & \textbf{Punkte lg \& an} & \textbf{Punkte sm \& so} \\
  \hline  		
  Keine Fremdwörter verwendet & \circletext{0} & \circletext{0} & \circletext{0} \\ 
  \hline
  Wenig Fremdwörter verwendet & 1 & 0 & 0 \\ 
  \hline
  Einige Fremdwörter verwendet & 2 & 0 & 0 \\
  \hline  
  Sehe viele Fremdwörter verwendet & 3 & 0 &  0 \\
  \hline  
\end{tabular}

\subsubsection{Wurde visuell gearbeitet? (Damit sind die Notizen inbegriffen)}
\begin{tabular}{| l | c | c | c |}
  \hline	
  \textbf{Ausprägung} & \textbf{Punkte lb \& lf} & \textbf{Punkte lg \& an} & \textbf{Punkte sm \& so} \\
  \hline  		
  Keine & 0 & 0 & 0 \\ 
  \hline
  Kaum & 1 & 1 & 1 \\ 
  \hline
  Einige & \circletext{2} & \circletext{2} & \circletext{2} \\
  \hline  
  Sehe viele & 3 & 3 & 3 \\
  \hline  
\end{tabular}

\subsubsection{Wurden die Erklärungen strukturiert?}
\begin{tabular}{| l | c | c | c |}
  \hline	
  \textbf{Ausprägung} & \textbf{Punkte lb \& lf} & \textbf{Punkte lg \& an} & \textbf{Punkte sm \& so} \\
  \hline  		
  Gar nich & 0  & 0 & 0 \\ 
  \hline
  Kaum & 1 & 1 & 1 \\ 
  \hline
  Genügend & 2 & 2 & 2 \\
  \hline  
  Sehr gute Struktur & \circletext{3} & \circletext{3} & \circletext{3} \\
  \hline  
\end{tabular}

\subsubsection{Wie viel wurde über den Prozess gesprochen? Wurde unnötig über Nebensächlichkeiten geredet?}
\begin{tabular}{| l | c | c | c |}
  \hline	
  \textbf{Ausprägung} & \textbf{Punkte lb \& lf} & \textbf{Punkte lg \& an} & \textbf{Punkte sm \& so} \\
  \hline  		
  Viele Nebensächlichkeiten & 0  & 0 & 0 \\ 
  \hline
  Einige Nebensächlichkeiten & 1 & 1 & 1 \\ 
  \hline
  Kaum Nebensächlichkeiten & 2 & 2 & 2 \\
  \hline  
  Keine Nebensächlichkeiten & \circletext{3} & \circletext{3} & \circletext{3} \\
  \hline  
\end{tabular}

\subsubsection{Wirkte die Person interessiert?}
\begin{tabular}{| l | c | c | c |}
  \hline	
  \textbf{Ausprägung} & \textbf{Punkte lb \& lf} & \textbf{Punkte lg \& an} & \textbf{Punkte sm \& so} \\
  \hline  		
  Gar nicht & 0  & 0 & 0 \\ 
  \hline
  Etwas & 1 & 0 & 0 \\ 
  \hline
  Interessiert & \circletext{2} & \circletext{0} & \circletext{0} \\
  \hline  
  Sehr Interessiert & 3 & 0 & 0 \\
  \hline  
\end{tabular}



\subsubsection{Summe der erreichten Punkte}
\begin{center}
  \begin{tabular}{ | p{7cm} | p{3cm} | p{3cm} |}
   \hline
   \textbf{Kernkompetenz} & \textbf{erreichte Punkte} & \textbf{maximale Punkte} \\ \hline
   Analytisches und systematisches Denken & 13 & 14\\ \hline
  Lernbereitschaft und Lernfähigkeit & 18 & 24\\ \hline
   Selbstmanagement und Selbstorganisation & 18 & 20\\ \hline
   \textbf{Total Übung 1} & \textbf{49} & \textbf{58}\\ \hline
  \end{tabular}
\end{center}

\subsubsection{Vorgehen des Kandidaten} \label{bemmat}

Matthias scrollte anfangs durch den Wikipedia Artikel und überflog den Text nur schnell. Er schien gemerkt zu haben, dass der Text relativ schwierig zu verstehen ist und suchte gleich eine andere Seite, wo alles viel einfacher und verständlicher erklärt wurde. Auf dieser Seite blieb er dann auch den Rest der Zeit. Dabei macht er sich stichwortartige Notizen.

\subsubsection{Präsentation}

Die Präsentation selbst war mit Abstand die Beste. Der Einstieg war sehr gelungen. Den Prozess konnte er sehr gut und verständlich erklären. Auch konnte er die Probleme und die Verwendung der Technik nachvollziehbar erläutern, was keiner der anderen Testkandidaten machen konnte. Jedoch blieb er jeglichen Fachbegriffen fern und erklärte alles in einfachen Worten. 


\subsection{Übung 2: Selbstmanagement und Selbstorganisation}
\subsubsection{Resultate der Bewertungskriterien}

\begin{center}
\vspace{0.2cm}
  \begin{longtable}{ | p{11cm} | p{2cm} |}
   \hline
   \textbf{macht sich der Proband Notizen?} & \textbf{Punkte} \\ \hline
   macht keine Notizen & 0 \\ \hline
   macht wenige Notizen & \circletext{1} \\ \hline
   macht viele Notizen & 2 \\ \hline
   notiert sich alles  & 3\\ \hline
   \textbf{schafft sich der Proband Übersicht?} & \textbf{Punkte} \\ \hline
   schafft sich keine Übersicht & 0 \\ \hline
   schafft sich fast keine Übersicht & 1 \\ \hline
   schafft sich Übersicht & \circletext{2} \\ \hline
   studiert die Aufgabenstellungen gründlich  & 3\\ \hline
   \textbf{erstellt der Proband eine Zeitplanung?} & \textbf{Punkte} \\ \hline
   erstellt keine Zeitplanung & 0 \\ \hline
   erstellt so etwas wie eine Zeitplanung & \circletext{1} \\ \hline
   erstellt eine gute Zeitplanung & 2 \\ \hline
   erstellt detail. Zeitplanung inkl. Reservezeiten  & 3\\ \hline
   \textbf{wendet der Proband das Eisenhower-Prinzip an?} & \textbf{Punkte} \\ \hline
   wendet Eisenhower-Prinzip nicht an & 0 \\ \hline
   versucht das Eisenhower-Prinzip anzuwenden & \circletext{1} \\ \hline
   wendet Eisenhower-Prinzip richtig an & 2 \\ \hline
   wendet Eisenhower-Prinzip korrekt an und leitet dadurch Konsequenzen für die Bearbeitung der Aufgaben ab  & 3\\ \hline
   \textbf{notiert sich der Proband Fragen?} & \textbf{Punkte} \\ \hline
   notiert keine Fragen & 0 \\ \hline
   notiert sich wenige relevante Fragen & \circletext{1} \\ \hline
   notiert einige relevante Fragen & 2 \\ \hline
   notiert und stellt alle relevanten Fragen  & 3\\ \hline
   \textbf{informiert der Proband über seinen Arbeitsstand?} & \textbf{Punkte} \\ \hline
   informiert den Prüfer nicht über den Arbeitsstand & 0 \\ \hline
   gibt fast keine Informationen weiter & \circletext{1} \\ \hline
   gibt viele Informationen an den Prüfer weiter & 2 \\ \hline
   gibt alle relevanten Informationen zeitgerecht an den Prüfer weiter & 3\\ \hline
%  \end{tabular}
%\end{center}
%\begin{center}
%  \begin{tabular}{ | p{11cm} | p{1cm} |}
%   \hline
   \textbf{Ausprägung analytisches Denken} & \textbf{Punkte} \\ \hline
   es fällt dem Geprüften sehr schwer die Aufgabenstellungen zu verstehen & 0 \\ \hline
   der Geprüfte bei einigen Aufgabenstellungen Mühe sie zu verstehen  & 1 \\ \hline
   der Geprüfte hat wenig Mühe die Aufgabenstellungen zu verstehen & \circletext{2} \\ \hline
   dem Geprüften waren alle Aufgabenstellungen sofort klar & 3\\ \hline
   \textbf{Ausprägung systematisches Denken} & \textbf{Punkte} \\ \hline
   der Geprüfte hat grosse Mühe, die Verknüpfungen zwischen den Aufgaben zu verstehen & 0 \\ \hline
   der Geprüfte hat teilweise Mühe, die Verknüpfungen zwischen den Aufgaben zu verstehen & \circletext{1} \\ \hline
   der Geprüfte findet und versteht die Abhängigkeiten der Aufgaben & 2 \\ \hline
   findet und versteht die Abhängigkeiten der Aufgaben auf Anhieb & 3\\ \hline
  \end{longtable}
\end{center}

\subsubsection{Summe der erreichten Punkte}

\begin{center}
  \begin{tabular}{ | p{7cm} | p{3cm} | p{3cm} |}
   \hline
   \textbf{Kernkompetenz} & \textbf{erreichte Punkte} & \textbf{maximale Punkte} \\ \hline
   Selbstmanagement und Selbstorganisation & 7 & 18\\ \hline
   Analytisches und systematisches Denken & 3 & 6\\ \hline
   \textbf{Total Übung 2} & \textbf{10} & \textbf{24}\\ \hline
  \end{tabular}
\end{center}

\subsubsection{Anmerkungen}
Da der Proband sich am Anfang der Übung eine ungenügende Übersicht verschafft hatte, merkte er erst etwa in der siebten Minute, dass er nicht genügend Angaben hatte, um die Transistorschaltung zu lösen. Weil ihm aber kein Fragezeitfenster mehr zur Verfügung stand, gab es für ihn keine Möglichkeit an diese Informationen zu kommen. Auch hat er in der Beantwortung des E-Mails angegeben er habe vier Minuten für die Korrektur des Messberichts aufgewendet, obwohl er diesen noch gar nicht angefangen hatte. Diese falsche Angabe hätte er auch umgehen können, wenn er am Anfang mehr Zeit aufgebracht hätte um die einzelnen Aufgaben zu studieren.