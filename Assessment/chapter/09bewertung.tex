\chapter{Bewertung Probanden}
\section{Einleitung}
Um die Tauglichkeit unserer ausgedachten Übungen, wie auch deren Schwierigkeitsgrad einschätzen zu können, wurde das Verhalten der Bewerber während dem  Assessment von uns beobachtet und mit den im Kapitel \ref{Assessment} auf Seite \pageref{Assessment} beschriebenen Bewertungskriterien eingeschätzt.\\
Die Bewertung wird unten, nach den geprüften Personen sortiert, aufgeführt.

\section{Tomo Bogdanovic}
\subsection{Übung 1: Logisches und analytisches Denken, Lernbereitschaft und Lernfähigkeit}

\subsection{Übung 2: Selbstmanagement und Selbstorganisation}
\subsubsection{Resultate der Bewertungskriterien}
\begin{center}
  \begin{tabular}{ | p{9cm} | p{1cm} |}
   \hline
   \textbf{macht sich der Proband Notizen?} & \textbf{Punkte} \\ \hline
   macht keine Notizen & 0 \\ \hline
   macht wenige Notizen & \circletext{1} \\ \hline
   macht viele Notizen & 2 \\ \hline
   notiert sich alles  & 3\\ \hline
   \textbf{schafft sich der Proband Übersicht?} & \textbf{Punkte} \\ \hline
   schafft sich keine Übersicht & 0 \\ \hline
   schafft sich fast keine Übersicht & 1 \\ \hline
   schafft sich Übersicht & 2 \\ \hline
   studiert die Aufgabenstellungen gründlich  & 3\\ \hline
   \textbf{erstellt der Proband eine Zeitplanung?} & \textbf{Punkte} \\ \hline
   erstellt keine Zeitplanung & 0 \\ \hline
   erstellt so etwas wie eine Zeitplanung & 1 \\ \hline
   erstellt eine gute Zeitplanung & 2 \\ \hline
   erstellt detail. Zeitplanung inkl. Reservezeiten  & 3\\ \hline
   \textbf{wendet der Proband das Eisenhower-Prinzip an?} & \textbf{Punkte} \\ \hline
   wendet Eisenhower-Prinzip nicht an & 0 \\ \hline
   versucht das Eisenhower-Prinzip anzuwenden & 1 \\ \hline
   wendet Eisenhower-Prinzip richtig an & 2 \\ \hline
   wendet Eisenhower-Prinzip korrekt an und leitet dadurch Konsequenzen für die Bearbeitung der Aufgaben ab  & 3\\ \hline
   \textbf{notiert sich der Proband Fragen?} & \textbf{Punkte} \\ \hline
   notiert keine Fragen & 0 \\ \hline
   notiert sich wenige relevante Fragen & 1 \\ \hline
   notiert einige relevante Fragen & 2 \\ \hline
   notiert und stellt alle relevanten Fragen  & 3\\ \hline
   \textbf{informiert der Proband über seinen Arbeitsstand?} & \textbf{Punkte} \\ \hline
   informiert den Prüfer nicht über den Arbeitsstand & 0 \\ \hline
   gibt fast keine Informationen weiter & 1 \\ \hline
   gibt viele Informationen an den Prüfer weiter & 2 \\ \hline
   gibt alle relevanten Informationen zeitgerecht an den Prüfer weiter & 3\\ \hline
  \end{tabular}
\end{center}
\begin{center}
  \begin{tabular}{ | p{11cm} | p{1cm} |}
   \hline
   \textbf{Ausprägung analytisches Denken} & \textbf{Punkte} \\ \hline
   es fällt dem Geprüften sehr schwer die Aufgabenstellungen zu verstehen & 0 \\ \hline
   der Geprüfte bei einigen Aufgabenstellungen Mühe sie zu verstehen  & 1 \\ \hline
   der Geprüfte hat wenig Mühe die Aufgabenstellungen zu verstehen & 2 \\ \hline
   dem Geprüften waren alle Aufgabenstellungen sofort klar & 3\\ \hline
   \textbf{Ausprägung systematisches Denken} & \textbf{Punkte} \\ \hline
   der Geprüfte hat grosse Mühe, die Verknüpfungen zwischen den Aufgaben zu verstehen & 0 \\ \hline
    der Geprüfte hat teilweise Mühe, die Verknüpfungen zwischen den Aufgaben zu verstehen & 1 \\ \hline
   der Geprüfte findet und versteht die Abhängigkeiten der Aufgaben & 2 \\ \hline
   findet und versteht die Abhängigkeiten der Aufgaben auf Anhieb & 3\\ \hline
  \end{tabular}
\end{center}

\subsubsection{Allgemeiner Eindruck}

\section{Michel André Nyffenegger}
\subsection{Übung 1: Logisches und analytisches Denken, Lernbereitschaft und Lernfähigkeit}

\subsection{Übung 2: Selbstmanagement und Selbstorganisation}
\subsubsection{Resultate der Bewertungskriterien}
\begin{center}
  \begin{tabular}{ | p{9cm} | p{1cm} |}
   \hline
   \textbf{macht sich der Proband Notizen?} & \textbf{Punkte} \\ \hline
   macht keine Notizen & 0 \\ \hline
   macht wenige Notizen & \circletext{1} \\ \hline
   macht viele Notizen & 2 \\ \hline
   notiert sich alles  & 3\\ \hline
   \textbf{schafft sich der Proband Übersicht?} & \textbf{Punkte} \\ \hline
   schafft sich keine Übersicht & 0 \\ \hline
   schafft sich fast keine Übersicht & 1 \\ \hline
   schafft sich Übersicht & 2 \\ \hline
   studiert die Aufgabenstellungen gründlich  & 3\\ \hline
   \textbf{erstellt der Proband eine Zeitplanung?} & \textbf{Punkte} \\ \hline
   erstellt keine Zeitplanung & 0 \\ \hline
   erstellt so etwas wie eine Zeitplanung & 1 \\ \hline
   erstellt eine gute Zeitplanung & 2 \\ \hline
   erstellt detail. Zeitplanung inkl. Reservezeiten  & 3\\ \hline
   \textbf{wendet der Proband das Eisenhower-Prinzip an?} & \textbf{Punkte} \\ \hline
   wendet Eisenhower-Prinzip nicht an & 0 \\ \hline
   versucht das Eisenhower-Prinzip anzuwenden & 1 \\ \hline
   wendet Eisenhower-Prinzip richtig an & 2 \\ \hline
   wendet Eisenhower-Prinzip korrekt an und leitet dadurch Konsequenzen für die Bearbeitung der Aufgaben ab  & 3\\ \hline
   \textbf{notiert sich der Proband Fragen?} & \textbf{Punkte} \\ \hline
   notiert keine Fragen & 0 \\ \hline
   notiert sich wenige relevante Fragen & 1 \\ \hline
   notiert einige relevante Fragen & 2 \\ \hline
   notiert und stellt alle relevanten Fragen  & 3\\ \hline
   \textbf{informiert der Proband über seinen Arbeitsstand?} & \textbf{Punkte} \\ \hline
   informiert den Prüfer nicht über den Arbeitsstand & 0 \\ \hline
   gibt fast keine Informationen weiter & 1 \\ \hline
   gibt viele Informationen an den Prüfer weiter & 2 \\ \hline
   gibt alle relevanten Informationen zeitgerecht an den Prüfer weiter & 3\\ \hline
  \end{tabular}
\end{center}
\begin{center}
  \begin{tabular}{ | p{11cm} | p{1cm} |}
   \hline
   \textbf{Ausprägung analytisches Denken} & \textbf{Punkte} \\ \hline
   es fällt dem Geprüften sehr schwer die Aufgabenstellungen zu verstehen & 0 \\ \hline
   der Geprüfte bei einigen Aufgabenstellungen Mühe sie zu verstehen  & 1 \\ \hline
   der Geprüfte hat wenig Mühe die Aufgabenstellungen zu verstehen & 2 \\ \hline
   dem Geprüften waren alle Aufgabenstellungen sofort klar & 3\\ \hline
   \textbf{Ausprägung systematisches Denken} & \textbf{Punkte} \\ \hline
   der Geprüfte hat grosse Mühe, die Verknüpfungen zwischen den Aufgaben zu verstehen & 0 \\ \hline
    der Geprüfte hat teilweise Mühe, die Verknüpfungen zwischen den Aufgaben zu verstehen & 1 \\ \hline
   der Geprüfte findet und versteht die Abhängigkeiten der Aufgaben & 2 \\ \hline
   findet und versteht die Abhängigkeiten der Aufgaben auf Anhieb & 3\\ \hline
  \end{tabular}
\end{center}

\subsubsection{Allgemeiner Eindruck}

\section{Matthias Schneider}
\subsection{Übung 1: Logisches und analytisches Denken, Lernbereitschaft und Lernfähigkeit}

\subsection{Übung 2: Selbstmanagement und Selbstorganisation}
\subsubsection{Resultate der Bewertungskriterien}
\begin{center}
  \begin{tabular}{ | p{9cm} | p{1cm} |}
   \hline
   \textbf{macht sich der Proband Notizen?} & \textbf{Punkte} \\ \hline
   macht keine Notizen & 0 \\ \hline
   macht wenige Notizen & \circletext{1} \\ \hline
   macht viele Notizen & 2 \\ \hline
   notiert sich alles  & 3\\ \hline
   \textbf{schafft sich der Proband Übersicht?} & \textbf{Punkte} \\ \hline
   schafft sich keine Übersicht & 0 \\ \hline
   schafft sich fast keine Übersicht & 1 \\ \hline
   schafft sich Übersicht & 2 \\ \hline
   studiert die Aufgabenstellungen gründlich  & 3\\ \hline
   \textbf{erstellt der Proband eine Zeitplanung?} & \textbf{Punkte} \\ \hline
   erstellt keine Zeitplanung & 0 \\ \hline
   erstellt so etwas wie eine Zeitplanung & 1 \\ \hline
   erstellt eine gute Zeitplanung & 2 \\ \hline
   erstellt detail. Zeitplanung inkl. Reservezeiten  & 3\\ \hline
   \textbf{wendet der Proband das Eisenhower-Prinzip an?} & \textbf{Punkte} \\ \hline
   wendet Eisenhower-Prinzip nicht an & 0 \\ \hline
   versucht das Eisenhower-Prinzip anzuwenden & 1 \\ \hline
   wendet Eisenhower-Prinzip richtig an & 2 \\ \hline
   wendet Eisenhower-Prinzip korrekt an und leitet dadurch Konsequenzen für die Bearbeitung der Aufgaben ab  & 3\\ \hline
   \textbf{notiert sich der Proband Fragen?} & \textbf{Punkte} \\ \hline
   notiert keine Fragen & 0 \\ \hline
   notiert sich wenige relevante Fragen & 1 \\ \hline
   notiert einige relevante Fragen & 2 \\ \hline
   notiert und stellt alle relevanten Fragen  & 3\\ \hline
   \textbf{informiert der Proband über seinen Arbeitsstand?} & \textbf{Punkte} \\ \hline
   informiert den Prüfer nicht über den Arbeitsstand & 0 \\ \hline
   gibt fast keine Informationen weiter & 1 \\ \hline
   gibt viele Informationen an den Prüfer weiter & 2 \\ \hline
   gibt alle relevanten Informationen zeitgerecht an den Prüfer weiter & 3\\ \hline
  \end{tabular}
\end{center}
\begin{center}
  \begin{tabular}{ | p{11cm} | p{1cm} |}
   \hline
   \textbf{Ausprägung analytisches Denken} & \textbf{Punkte} \\ \hline
   es fällt dem Geprüften sehr schwer die Aufgabenstellungen zu verstehen & 0 \\ \hline
   der Geprüfte bei einigen Aufgabenstellungen Mühe sie zu verstehen  & 1 \\ \hline
   der Geprüfte hat wenig Mühe die Aufgabenstellungen zu verstehen & 2 \\ \hline
   dem Geprüften waren alle Aufgabenstellungen sofort klar & 3\\ \hline
   \textbf{Ausprägung systematisches Denken} & \textbf{Punkte} \\ \hline
   der Geprüfte hat grosse Mühe, die Verknüpfungen zwischen den Aufgaben zu verstehen & 0 \\ \hline
    der Geprüfte hat teilweise Mühe, die Verknüpfungen zwischen den Aufgaben zu verstehen & 1 \\ \hline
   der Geprüfte findet und versteht die Abhängigkeiten der Aufgaben & 2 \\ \hline
   findet und versteht die Abhängigkeiten der Aufgaben auf Anhieb & 3\\ \hline
  \end{tabular}
\end{center}

\subsubsection{Allgemeiner Eindruck}