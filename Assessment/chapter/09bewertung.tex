\chapter{Bewertung Probanden}
\section{Einleitung}
Um die Tauglichkeit unserer ausgedachten Übungen, wie auch deren Schwierigkeitsgrad einschätzen zu können, wurde das Verhalten der Bewerber während dem  Assessment von uns beobachtet und mit den im Kapitel \ref{Assessment} auf Seite \pageref{Assessment} beschriebenen Bewertungskriterien eingeschätzt.\\
Die Bewertung wird unten, nach den geprüften Personen sortiert, aufgeführt.

\section{Tomo Bogdanovic}
\subsection{Übung 1: Analytisches und systematisches Denken, Lernbereitschaft und Lernfähigkeit}
\subsubsection{Summe der erreichten Punkte}
\begin{center}
  \begin{tabular}{ | p{7cm} | p{3cm} | p{3cm} |}
   \hline
   \textbf{Kernkompetenz} & \textbf{erreichte Punkte} & \textbf{maximale Punkte} \\ \hline
   Analytisches und systematisches Denken & x & x\\ \hline
  Lernbereitschaft und Lernfähigkeit & x & x\\ \hline
   \textbf{Total Übung 1} & \textbf{x} & \textbf{x}\\ \hline
  \end{tabular}
\end{center}
\subsection{Übung 2: Selbstmanagement und Selbstorganisation}
\subsubsection{Resultate der Bewertungskriterien}
\begin{center}
  \begin{tabular}{ | p{9cm} | p{1cm} |}
   \hline
   \textbf{macht sich der Proband Notizen?} & \textbf{Punkte} \\ \hline
   macht keine Notizen & 0 \\ \hline
   macht wenige Notizen & 1 \\ \hline
   macht viele Notizen & \circletext{2} \\ \hline
   notiert sich alles  & 3\\ \hline
   \textbf{schafft sich der Proband Übersicht?} & \textbf{Punkte} \\ \hline
   schafft sich keine Übersicht & 0 \\ \hline
   schafft sich fast keine Übersicht & 1 \\ \hline
   schafft sich Übersicht & \circletext{2} \\ \hline
   studiert die Aufgabenstellungen gründlich  & 3\\ \hline
   \textbf{erstellt der Proband eine Zeitplanung?} & \textbf{Punkte} \\ \hline
   erstellt keine Zeitplanung & 0 \\ \hline
   erstellt so etwas wie eine Zeitplanung & \circletext{1} \\ \hline
   erstellt eine gute Zeitplanung & 2 \\ \hline
   erstellt detail. Zeitplanung inkl. Reservezeiten  & 3\\ \hline
   \textbf{wendet der Proband das Eisenhower-Prinzip an?} & \textbf{Punkte} \\ \hline
   wendet Eisenhower-Prinzip nicht an & 0 \\ \hline
   versucht das Eisenhower-Prinzip anzuwenden & \circletext{1} \\ \hline
   wendet Eisenhower-Prinzip richtig an & 2 \\ \hline
   wendet Eisenhower-Prinzip korrekt an und leitet dadurch Konsequenzen für die Bearbeitung der Aufgaben ab  & 3\\ \hline
   \textbf{notiert sich der Proband Fragen?} & \textbf{Punkte} \\ \hline
   notiert keine Fragen & \circletext{0} \\ \hline
   notiert sich wenige relevante Fragen & 1 \\ \hline
   notiert einige relevante Fragen & 2 \\ \hline
   notiert und stellt alle relevanten Fragen  & 3\\ \hline
   \textbf{informiert der Proband über seinen Arbeitsstand?} & \textbf{Punkte} \\ \hline
   informiert den Prüfer nicht über den Arbeitsstand & \circletext{0} \\ \hline
   gibt fast keine Informationen weiter & 1 \\ \hline
   gibt viele Informationen an den Prüfer weiter & 2 \\ \hline
   gibt alle relevanten Informationen zeitgerecht an den Prüfer weiter & 3\\ \hline
  \end{tabular}
\end{center}
\begin{center}
  \begin{tabular}{ | p{11cm} | p{1cm} |}
   \hline
   \textbf{Ausprägung analytisches Denken} & \textbf{Punkte} \\ \hline
   es fällt dem Geprüften sehr schwer die Aufgabenstellungen zu verstehen & 0 \\ \hline
   der Geprüfte bei einigen Aufgabenstellungen Mühe sie zu verstehen  & 1 \\ \hline
   der Geprüfte hat wenig Mühe die Aufgabenstellungen zu verstehen & \circletext{2} \\ \hline
   dem Geprüften waren alle Aufgabenstellungen sofort klar & 3\\ \hline
   \textbf{Ausprägung systematisches Denken} & \textbf{Punkte} \\ \hline
   der Geprüfte hat grosse Mühe, die Verknüpfungen zwischen den Aufgaben zu verstehen & 0 \\ \hline
    der Geprüfte hat teilweise Mühe, die Verknüpfungen zwischen den Aufgaben zu verstehen & 1 \\ \hline
   der Geprüfte findet und versteht die Abhängigkeiten der Aufgaben & \circletext{2} \\ \hline
   findet und versteht die Abhängigkeiten der Aufgaben auf Anhieb & 3\\ \hline
  \end{tabular}
\end{center}

\subsubsection{Summe der erreichten Punkte}

\begin{center}
  \begin{tabular}{ | p{7cm} | p{3cm} | p{3cm} |}
   \hline
   \textbf{Kernkompetenz} & \textbf{erreichte Punkte} & \textbf{maximale Punkte} \\ \hline
   Selbstmanagement und Selbstorganisation & 6 & 18\\ \hline
   Analytisches und systematisches Denken & 4 & 6\\ \hline
   \textbf{Total Übung 2} & \textbf{10} & \textbf{24}\\ \hline
  \end{tabular}
\end{center}

\subsubsection{Anmerkungen}
Obwohl dem Probanden deutlich mitgeteilt wurde, dass Fragen erlaubt sind und auch einige Unklarheiten herrschten, vergass er, fragen zu stellen. Als ich ihm im Feedback-Gespräch mitteilte, dass die Aufgaben gar nicht ohne Nachfragen beim Prüfer zu lösen seien, zeigte er sich beschämt, dass er keine Fragen gestellt hatte.

\section{Michel André Nyffenegger}
\subsection{Übung 1: Analytisches und systematisches Denken, Lernbereitschaft und Lernfähigkeit}
\subsubsection{Summe der erreichten Punkte}
\begin{center}
  \begin{tabular}{ | p{7cm} | p{3cm} | p{3cm} |}
   \hline
   \textbf{Kernkompetenz} & \textbf{erreichte Punkte} & \textbf{maximale Punkte} \\ \hline
   Analytisches und systematisches Denken & x & x\\ \hline
  Lernbereitschaft und Lernfähigkeit & x & x\\ \hline
   \textbf{Total Übung 1} & \textbf{x} & \textbf{x}\\ \hline
  \end{tabular}
\end{center}
\subsection{Übung 2: Selbstmanagement und Selbstorganisation}
\subsubsection{Resultate der Bewertungskriterien}
\begin{center}
  \begin{tabular}{ | p{9cm} | p{1cm} |}
   \hline
   \textbf{macht sich der Proband Notizen?} & \textbf{Punkte} \\ \hline
   macht keine Notizen & 0 \\ \hline
   macht wenige Notizen & \circletext{1} \\ \hline
   macht viele Notizen & 2 \\ \hline
   notiert sich alles  & 3\\ \hline
   \textbf{schafft sich der Proband Übersicht?} & \textbf{Punkte} \\ \hline
   schafft sich keine Übersicht & 0 \\ \hline
   schafft sich fast keine Übersicht & \circletext{1} \\ \hline
   schafft sich Übersicht & 2 \\ \hline
   studiert die Aufgabenstellungen gründlich  & 3\\ \hline
   \textbf{erstellt der Proband eine Zeitplanung?} & \textbf{Punkte} \\ \hline
   erstellt keine Zeitplanung & 0 \\ \hline
   erstellt so etwas wie eine Zeitplanung & \circletext{1} \\ \hline
   erstellt eine gute Zeitplanung & 2 \\ \hline
   erstellt detail. Zeitplanung inkl. Reservezeiten  & 3\\ \hline
   \textbf{wendet der Proband das Eisenhower-Prinzip an?} & \textbf{Punkte} \\ \hline
   wendet Eisenhower-Prinzip nicht an & 0 \\ \hline
   versucht das Eisenhower-Prinzip anzuwenden & 1 \\ \hline
   wendet Eisenhower-Prinzip richtig an & \circletext{2} \\ \hline
   wendet Eisenhower-Prinzip korrekt an und leitet dadurch Konsequenzen für die Bearbeitung der Aufgaben ab  & 3\\ \hline
   \textbf{notiert sich der Proband Fragen?} & \textbf{Punkte} \\ \hline
   notiert keine Fragen & \circletext{0} \\ \hline
   notiert sich wenige relevante Fragen & 1 \\ \hline
   notiert einige relevante Fragen & 2 \\ \hline
   notiert und stellt alle relevanten Fragen  & 3\\ \hline
   \textbf{informiert der Proband über seinen Arbeitsstand?} & \textbf{Punkte} \\ \hline
   informiert den Prüfer nicht über den Arbeitsstand & 0 \\ \hline
   gibt fast keine Informationen weiter & \circletext{1} \\ \hline
   gibt viele Informationen an den Prüfer weiter & 2 \\ \hline
   gibt alle relevanten Informationen zeitgerecht an den Prüfer weiter & 3\\ \hline
  \end{tabular}
\end{center}
\begin{center}
  \begin{tabular}{ | p{11cm} | p{1cm} |}
   \hline
   \textbf{Ausprägung analytisches Denken} & \textbf{Punkte} \\ \hline
   es fällt dem Geprüften sehr schwer die Aufgabenstellungen zu verstehen & 0 \\ \hline
   der Geprüfte bei einigen Aufgabenstellungen Mühe sie zu verstehen  & 1 \\ \hline
   der Geprüfte hat wenig Mühe die Aufgabenstellungen zu verstehen & \circletext{2} \\ \hline
   dem Geprüften waren alle Aufgabenstellungen sofort klar & 3\\ \hline
   \textbf{Ausprägung systematisches Denken} & \textbf{Punkte} \\ \hline
   der Geprüfte hat grosse Mühe, die Verknüpfungen zwischen den Aufgaben zu verstehen & 0 \\ \hline
    der Geprüfte hat teilweise Mühe, die Verknüpfungen zwischen den Aufgaben zu verstehen & \circletext{1} \\ \hline
   der Geprüfte findet und versteht die Abhängigkeiten der Aufgaben & 2 \\ \hline
   findet und versteht die Abhängigkeiten der Aufgaben auf Anhieb & 3\\ \hline
  \end{tabular}
\end{center}

\subsubsection{Summe der erreichten Punkte}

\begin{center}
  \begin{tabular}{ | p{7cm} | p{3cm} | p{3cm} |}
   \hline
   \textbf{Kernkompetenz} & \textbf{erreichte Punkte} & \textbf{maximale Punkte} \\ \hline
   Selbstmanagement und Selbstorganisation & 6 & 18\\ \hline
   Analytisches und systematisches Denken & 3 & 6\\ \hline
   \textbf{Total Übung 2} & \textbf{9} & \textbf{24}\\ \hline
  \end{tabular}
\end{center}

\subsubsection{Anmerkungen}
Der Proband hat frühzeitig gemerkt, dass er für das vollständige Lösen der Emitterschaltung mehr Angaben braucht. Trotzdem hat er dazu nicht in eine Frage gestellt, sonst hätte der Prüfer ihm das Datenblatt des Transistors ausgehändigt, auf dem alle benötigten Angaben aufgeführt sind.
\section{Matthias Schneider}
\subsection{Übung 1:Analytisches und systematisches Denken, Lernbereitschaft und Lernfähigkeit}
\subsubsection{Summe der erreichten Punkte}
\begin{center}
  \begin{tabular}{ | p{7cm} | p{3cm} | p{3cm} |}
   \hline
   \textbf{Kernkompetenz} & \textbf{erreichte Punkte} & \textbf{maximale Punkte} \\ \hline
   Analytisches und systematisches Denken & x & x\\ \hline
  Lernbereitschaft und Lernfähigkeit & x & x\\ \hline
   \textbf{Total Übung 1} & \textbf{x} & \textbf{x}\\ \hline
  \end{tabular}
\end{center}
\subsection{Übung 2: Selbstmanagement und Selbstorganisation}
\subsubsection{Resultate der Bewertungskriterien}
\begin{center}
  \begin{tabular}{ | p{9cm} | p{1cm} |}
   \hline
   \textbf{macht sich der Proband Notizen?} & \textbf{Punkte} \\ \hline
   macht keine Notizen & 0 \\ \hline
   macht wenige Notizen & \circletext{1} \\ \hline
   macht viele Notizen & 2 \\ \hline
   notiert sich alles  & 3\\ \hline
   \textbf{schafft sich der Proband Übersicht?} & \textbf{Punkte} \\ \hline
   schafft sich keine Übersicht & 0 \\ \hline
   schafft sich fast keine Übersicht & 1 \\ \hline
   schafft sich Übersicht & \circletext{2} \\ \hline
   studiert die Aufgabenstellungen gründlich  & 3\\ \hline
   \textbf{erstellt der Proband eine Zeitplanung?} & \textbf{Punkte} \\ \hline
   erstellt keine Zeitplanung & 0 \\ \hline
   erstellt so etwas wie eine Zeitplanung & \circletext{1} \\ \hline
   erstellt eine gute Zeitplanung & 2 \\ \hline
   erstellt detail. Zeitplanung inkl. Reservezeiten  & 3\\ \hline
   \textbf{wendet der Proband das Eisenhower-Prinzip an?} & \textbf{Punkte} \\ \hline
   wendet Eisenhower-Prinzip nicht an & 0 \\ \hline
   versucht das Eisenhower-Prinzip anzuwenden & \circletext{1} \\ \hline
   wendet Eisenhower-Prinzip richtig an & 2 \\ \hline
   wendet Eisenhower-Prinzip korrekt an und leitet dadurch Konsequenzen für die Bearbeitung der Aufgaben ab  & 3\\ \hline
   \textbf{notiert sich der Proband Fragen?} & \textbf{Punkte} \\ \hline
   notiert keine Fragen & 0 \\ \hline
   notiert sich wenige relevante Fragen & \circletext{1} \\ \hline
   notiert einige relevante Fragen & 2 \\ \hline
   notiert und stellt alle relevanten Fragen  & 3\\ \hline
   \textbf{informiert der Proband über seinen Arbeitsstand?} & \textbf{Punkte} \\ \hline
   informiert den Prüfer nicht über den Arbeitsstand & 0 \\ \hline
   gibt fast keine Informationen weiter & \circletext{1} \\ \hline
   gibt viele Informationen an den Prüfer weiter & 2 \\ \hline
   gibt alle relevanten Informationen zeitgerecht an den Prüfer weiter & 3\\ \hline
  \end{tabular}
\end{center}
\begin{center}
  \begin{tabular}{ | p{11cm} | p{1cm} |}
   \hline
   \textbf{Ausprägung analytisches Denken} & \textbf{Punkte} \\ \hline
   es fällt dem Geprüften sehr schwer die Aufgabenstellungen zu verstehen & 0 \\ \hline
   der Geprüfte bei einigen Aufgabenstellungen Mühe sie zu verstehen  & 1 \\ \hline
   der Geprüfte hat wenig Mühe die Aufgabenstellungen zu verstehen & \circletext{2} \\ \hline
   dem Geprüften waren alle Aufgabenstellungen sofort klar & 3\\ \hline
   \textbf{Ausprägung systematisches Denken} & \textbf{Punkte} \\ \hline
   der Geprüfte hat grosse Mühe, die Verknüpfungen zwischen den Aufgaben zu verstehen & 0 \\ \hline
   der Geprüfte hat teilweise Mühe, die Verknüpfungen zwischen den Aufgaben zu verstehen & \circletext{1} \\ \hline
   der Geprüfte findet und versteht die Abhängigkeiten der Aufgaben & 2 \\ \hline
   findet und versteht die Abhängigkeiten der Aufgaben auf Anhieb & 3\\ \hline
  \end{tabular}
\end{center}

\subsubsection{Summe der erreichten Punkte}

\begin{center}
  \begin{tabular}{ | p{7cm} | p{3cm} | p{3cm} |}
   \hline
   \textbf{Kernkompetenz} & \textbf{erreichte Punkte} & \textbf{maximale Punkte} \\ \hline
   Selbstmanagement und Selbstorganisation & 7 & 18\\ \hline
   Analytisches und systematisches Denken & 3 & 6\\ \hline
   \textbf{Total Übung 2} & \textbf{10} & \textbf{24}\\ \hline
  \end{tabular}
\end{center}

\subsubsection{Anmerkungen}
Da der Proband sich nicht eine genügende Übersicht verschafft hatte am Anfang der Übung, merkte er erst etwa in der 17 Minute, dass er nicht genügend Angaben hatte um die Transistorschaltung zu lösen. Weil ihm aber kein Fragezeitfenster mehr zur Verfügung stand, gab es für ihn keine Möglichkeit an diese Informationen zu kommen. Auch hat er in der Beantwortung des E-Mails angegeben er habe vier Minuten für die Korrektur des Messberichts aufgewendet, obwohl er diesen noch gar nicht angefangen hatte. Diese falsche Angabe hätte er auch umgehen können, wenn er am Anfang mehr Zeit aufgebracht hätte um die einzelnen Aufgaben zu studieren.