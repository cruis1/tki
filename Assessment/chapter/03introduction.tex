%03introduction.tex

\chapter{Einleitung}

Das Assessment wird heutzutage vielerorts verwendet, um die Personalauswahl und Personalentwicklung zu unterstützen. Hierbei geht es darum, nachvollziehbare und überprüfbare Kriterien aufzustellen, womit eine Person möglichst objektiv bewertet werden kann. Somit ist das Assessment auch ein geeignetes Tool, um das Unternehmen vor Klagen wegen Nichtbeachtung des Gleichheitsgrundsatzes zu schützen. \\ 
Diese Tests werden in der Regel durch geschulte Beobachter durchgeführt. Grosskonzerne verfügen hier eigens dafür angestellte Psychologen. \\
In diesem Projektbericht geht es darum, mittels Umfragen die wichtigsten Kernkompetenzen von Elektroingenieuren zu ermitteln und anschliessend zwei Tests in Form eines Assessments zu schreiben, die das Vorhandensein der Kernkompetenzen möglichst gut überprüfen können. \\
Das entwickelte Assessmentwerkzeug soll auf Tauglichkeit geprüft und diese ausgewertet werden. Dies geschieht, indem das Verfahren mit verschiedenen Probanden durchgeführt wird. Zu Dokumentations- und Auswertungszwecken werden die Verfahren per Video aufgezeichnet. So kann das Werkzeug zum Schluss präzise ausgewertet und Verbesserungsmöglichkeiten formuliert werden.
  

