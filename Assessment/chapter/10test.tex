\chapter{Auswertung des Assessments}

Im anschliessenden Abschnitt wird die Tauglichkeit und Einsatzfähigkeit des in den vorhergehenden Kapiteln beschriebenen Assessmentverfahrens ausgewertet. Dazu wird zuerst auf die zwei Übungen im Speziellen und dann auf das ganze Assessment eingegangen. Es wird versucht, Mängel festzustellen und Verbesserungsvorschläge einzubringen.

\section{Auswertung Übung 1}
\section{Auswertung Übung 2}
\subsection{Tiefe erreichte Werte}
Anhand der drei bisher durchgeführten Assessments lässt sich natürlich noch keine statistische Aussage machen, trotzdem sind Tendenzen zu beobachten. So fällt zum Beispiel auf, dass die Probanden bei der Kernkompetenz \textit{Selbstmanagement und Selbstorganisation} im Schnitt nur 35 Prozent der Punkte erreicht haben.
\subsubsection{Mögliche Massnahmen}
\begin{itemize}
\item Die Punkteausbeute im unteren Drittel wird als ausreichend befunden. Es wird einfach angenommen, dass mögliche exzellente Kandidaten Punkte im mittleren bis oberen Drittel erreichen können. 
\item 
\end{itemize}
Um verifizieren zu können, dass die erreichten Punkte im Durchschnitt wirklich nicht höher liegen, müsste eine statistisch akzeptable Anzahl Proben verfügbar sein. Dies ist jedoch vom zeitlichen Aufwand her (siehe Punkt \ref{ssec:Zeitverhaeltnisse} auf Seite \pageref{ssec:Zeitverhaeltnisse}) nicht mehr im Rahmen der Aufgabenstellung. 
\subsection{Punkteverteilung für Kernkompetenzen}
Für die Kernkompetenz \textit{Analytisches und systematisches Denken} werden in dieser Übung maximal sechs Punkte vergeben, wobei für \textit{Selbstmanagement und Selbstorganisation} dreimal so viele erreicht werden können. Diese Übung ist also zu drei Vierteln auf Selbstorganisation und nur zu einem viertel auf \textit{Analytisches und systematisches Denken} ausgelegt.
\subsubsection{Mögliche Massnahmen}
\section{Assessment allgemein}
\subsection{Schwergewichtsbildung Kernkompetenzen}
\subsection{Analytisches und systematisches Denken}