\chapter{Auswertung des Assessments}

Im anschliessenden Abschnitt wird die Tauglichkeit und Einsatzfähigkeit des in den vorhergehenden Kapiteln beschriebenen Assessmentverfahrens ausgewertet. Dazu wird zuerst auf die zwei Übungen im Speziellen und dann auf das ganze Assessment eingegangen. Es wird versucht, Mängel festzustellen und Verbesserungsvorschläge einzubringen.

\section{Auswertung Übung 1}

\subsection{Erreichte Punkte}
Hier werden die erreichten Punkte nochmals kurz dargestellt.
\subsubsection{Tomo Bogdanovic}

\subsubsection{Summe der erreichten Punkte}
\begin{center}
  \begin{tabular}{ | p{7cm} | p{3cm} | p{3cm} |}
   \hline
   \textbf{Kernkompetenz} & \textbf{erreichte Punkte} & \textbf{maximale Punkte} \\ \hline
   Analytisches und systematisches Denken & 11 & 14\\ \hline
  Lernbereitschaft und Lernfähigkeit & 19 & 24\\ \hline
   Selbstmanagement und Selbstorganisation & 15 & 20\\ \hline
   \textbf{Total Übung 1} & \textbf{45} & \textbf{58}\\ \hline
  \end{tabular}
\end{center}

\subsubsection{Michel Andr\'{e} Nyffenegger}

\subsubsection{Summe der erreichten Punkte}
\begin{center}
  \begin{tabular}{ | p{7cm} | p{3cm} | p{3cm} |}
   \hline
   \textbf{Kernkompetenz} & \textbf{erreichte Punkte} & \textbf{maximale Punkte} \\ \hline
   Analytisches und systematisches Denken & 7 & 14\\ \hline
  Lernbereitschaft und Lernfähigkeit & 12 & 24\\ \hline
   Selbstmanagement und Selbstorganisation & 8 & 20\\ \hline
   \textbf{Total Übung 1} & \textbf{27} & \textbf{58}\\ \hline
  \end{tabular}
\end{center}

\subsubsection{Matthias Schneider}

\subsubsection{Summe der erreichten Punkte}
\begin{center}
  \begin{tabular}{ | p{7cm} | p{3cm} | p{3cm} |}
   \hline
   \textbf{Kernkompetenz} & \textbf{erreichte Punkte} & \textbf{maximale Punkte} \\ \hline
   Analytisches und systematisches Denken & 13 & 14\\ \hline
  Lernbereitschaft und Lernfähigkeit & 18 & 24\\ \hline
   Selbstmanagement und Selbstorganisation & 18 & 20\\ \hline
   \textbf{Total Übung 1} & \textbf{49} & \textbf{58}\\ \hline
  \end{tabular}
\end{center}

\subsection{Vergleiche unter den Testkandidaten}
Keines der Kandidaten hatte vorher signifikantes Vorwissen, was auch ein Ziel im Design des Tests war.
Dass Michel gemäss der Beschreibung im Teil (\ref{bemmic}) mit Abstand am schlechtesten abschneidet war zu erwarten. In allen drei Kernkompetenzen schneidet er mit der hälfte oder weniger Punkten ab. Bei den beiden anderen Testkandidaten sind die Resultate viel interessanter. \\
Während Tomo bei der Lernbereitschaft und Lernfähigkeit einen Punkt mehr sammeln konnte als Matthias, konnte sich Matthias bei den Kernkompetenzen Analytisches/Systematisches Denken und Selbstmanagement/Selbstorganisation um zwei bzw. drei Punkte von Tomo abheben. Dieses Resultat deckt sich sehr gut mit den Beobachtungen. \\
Tomo ging viel offener und interessierter an das Thema heran. Zudem liess sich Tomo in den mit vielen Fachbegriffen gespickten Wikipedia-Artikel ein, was dazu führte, dass er viele Fachbegriffe erklären konnte. Durch diese beiden Parameter konnte sich Tomo gesammthaft einen Punktevorteil in der Lernbereitschaft ergattern. Der Punktevorteil ist jedoch marginal und kaum relevant. In allen anderen Punkten ist Matthias klar im Vorteil. Matthias überzeugte durch seine speditive Herangehensweise. Bei ihm ist das Verständnis zum bearbeiteten Thema gemäss der Beschreibung (\ref{bemmat}) hervorragend. 

\section{Auswertung Übung 2}

\subsection{Tiefe erreichte Punktewerte}

Anhand der drei bisher durchgeführten Assessments lässt sich natürlich noch keine statistische Aussage machen, trotzdem sind Tendenzen zu beobachten. So fällt zum Beispiel auf, dass die Probanden bei der Kernkompetenz \textit{Selbstmanagement und Selbstorganisation} im Schnitt nur 35 Prozent der Punkte erreicht haben.\\
Um verifizieren zu können, dass die erreichten Punkte im Durchschnitt wirklich nicht höher liegen, müsste eine statistisch akzeptable Anzahl Proben verfügbar sein. Dies liegt jedoch vom zeitlichen Aufwand her (siehe Punkt \ref{ssec:Zeitverhaeltnisse} auf Seite \pageref{ssec:Zeitverhaeltnisse}) nicht mehr im Rahmen der Aufgabenstellung.
\subsubsection{Mögliche Massnahmen}
Um der niedrigen Punkteausbeute entgegenzuwirken bieten sich folgende Massnahmen an:
\begin{itemize}
\item Das zur Bearbeitung der Übung zur Verfügung gestellte Zeitfenster wird erweitert. Die Nachteile dieser Variante sind aber:
\begin{enumerate}
\item Die gesamte Zeit um das Assessment durchzuführen steigt pro Bewerber an
\item Die Auswertung wird schwieriger, da mit dem wegfallenden Zeitdruck die Leistungen der Bewerber sich angleichen würden
\end{enumerate}
\item Die Bewertungsskala wird angepasst. Dies bedeutet das die Leistungen der Bewerber weniger streng bewertet werden
\item Die Punkteausbeute im unteren Drittel wird als ausreichend befunden. Es wird einfach angenommen, dass mögliche exzellente Kandidaten Punkte im mittleren bis oberen Drittel erreichen können
\end{itemize} 
\subsection{Punkteverteilung für Kernkompetenzen}
Für die Kernkompetenz \textit{Analytisches und systematisches Denken} werden in dieser Übung maximal sechs Punkte vergeben, wobei für \textit{Selbstmanagement und Selbstorganisation} dreimal so viele erreicht werden können. Diese Übung ist also zu drei Vierteln auf Selbstorganisation und nur zu einem viertel auf \textit{Analytisches und systematisches Denken} ausgelegt.
\subsubsection{Mögliche Massnahmen}
\begin{itemize}
\item Es wird versucht mehr Elemente der entsprechenden Kernkompetenz in die Übung einfliessen zu lassen
\item Es wird eine separate Übung für das Testen der Kompetenz \textit{Analytisches und systematisches Denken} erstellt
\end{itemize}


\section{Assessment allgemein}
\subsection{Schwergewichtsbildung Kernkompetenzen}
\begin{center}
  \begin{tabular}{ | p{7cm} | p{3cm} | p{3cm} |}
   \hline
   \textbf{Kernkompetenz} & \textbf{durchschnittlich erreichte Punkte} & \textbf{maximal erreichbare Punkte} \\ \hline
   Lernbereitschaft und Lernfähigkeit & 16 & 24\\ \hline
   Analytisches und systematisches Denken & 13 & 19\\ \hline
   Selbstmanagement und Selbstorganisation & 20 & 36\\ \hline
  \end{tabular}
\end{center}
Wie anhand der in obiger Tabelle zusammengefasster Punkteverteilung der einzelnen Kernkompetenzen ersichtlich, werden diese nicht so gewertet wie in der Grafik \ref{fig:auswerkomp} auf Seite \pageref{fig:auswerkomp} aufgeführt. Die Kernkompetenz \textit{Analytisches und systematisches Denken} wird am schwächsten gewichtet, obschon sie zusammen mit \textit{Lernbereitschaft und Lernfähigkeit} von den befragten Ingenieuren als am wichtigsten eingestuft wurde.
\subsubsection{Mögliche Verbesserungsmassnahmen}
\begin{itemize}
\item Es wird eine separate Übung für das Testen der Kompetenz \textit{Analytisches und systematisches Denken} erstellt
\end{itemize}
\subsection{Problem Vorwissen bei Übung 1}
