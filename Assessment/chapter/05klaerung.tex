%04chapterFremdeinsch.tex
\chapter{Klärung der Aufgabenstellung}

Anfangs war nicht ganz klar, wie alles von statten gehen sollte. Aus diesem Grunde haben wir uns Fragen aufgeschrieben und alle gleich in einer Sitzung mit dem Dozenten besprochen. Folgendes kam dabei raus:

\textbf{1. Was ist genau der Umfang des Assessmentberichtes?} \\
Antwort: Es wurde keine konkrete Zahl genannt. Im Umfang sind wir freigestellt.

\textbf{2. Welche Vorlage sind vorhanden?} \\
Antwort: Alles ist im Moodle vorhanden. Es wird verlangt, dass wir uns selbstständig informieren und alle Dokumente durchforsten.

\textbf{3. Muss Projektplanung mit MS Project gemacht werden? Gibt es Alternativen?}\\
Antwort: MS Project wird verlangt.

\textbf{Wird Auftrag noch spezifiziert oder müssen wir mit dem arbeiten was auf Doodle ist?} \\
Antwort: Es wird keine weitere Spezifizierungen geben.

\textbf{5. Wie muss mit allen erstellten Dokumenten verfahren werden?} \\
Antwort: Sitzungsprotokolle und Traktandenlisten usw. kommen als Anhang in den Projektbericht.

