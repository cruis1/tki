\chapter{Projektauftrag}
Die nachfolgende Analyse des Projektauftrags wurde anhand der Vorlage \textit{HSR\_Projektauftrag.docx} vorgenommen. Einige Punkte mögen daher unpassend erscheinen, wurden aber so gut wie möglich auf unser Projekt angewandt.
\section*{Assessment Center}

\section{Ausgangssituation und Kontext}

\subsection{Ist Situation}

\subsection{Problem}

\subsection{Rahmenbedingungen}

\subsection{Interessensvertreter/Stakeholder}

Da wir in unserem Projekt keine Firma als Projektpartner haben, ist neben uns der einzige Stakeholder der Dozent. 

\section{Projektziele}

\subsection{Übergeordnete Ziele, Projektgesamtziel}

\subsection{Systemziele}

\subsection{Abwicklungsziele}

\subsection{Abgrenzung}


\section{Risiken und Chancen}

\subsection{Risiken}

\subsection{Chancen}

\subsection{Kritische Ergolgsfaktoren}


\section{Probemlösung}

\subsection{Lösungsvarianten}

\subsection{Lösungsauswahl}


\section{Grobplanung}

\subsection{Rahmen}

\begin{tabular}{ | l | l | l | l |}
    \hline
    Projektbeginn: & blabla & Projektende: & balbla \\ \hline
\end{tabular}

\subsection{Meilensteine}

\begin{tabular}{ | l | l | l |}
    \hline
    Was? & Wer? & Ca. wann? \\ \hline
    blbal & blbal & blbal \\ \hline
    blbal & blbal & blbal \\ \hline
    blbal & blbal & blbal \\ \hline
\end{tabular}

\section{Ressourcen/Kosten}

\section{Projektorganisation}

\section{Machbarkeit}

\section{Unterschriften}