\chapter{Analyse Projektauftrag}
Die nachfolgende Analyse des Projektauftrags wurde anhand der Vorlage \textit{HSR\_Projektauftrag.docx} vorgenommen. Einige Punkte mögen daher eventuell unpassend erscheinen, wurden aber so gut wie möglich auf unser Projekt angewandt.
\section*{Assessment Center}

\section{Ausgangssituation und Kontext}
Der Stoff des Moduls Teamkommunikation für Ingenieure soll nicht nur theoretisch vermittelt, sondern auch praktisch erfahren werden können. Dazu wird ein industrienahes Projekt als Turngerät verwendet. Anhand von diesem Projekt sollen möglichst viele praktische Erfahrungen zum Thema Teamkommunikation gemacht werden. Das Turngerät ist in unserem Falle der Auftrag Assessment Center.

\subsection{Ist Situation}
Das Projekt Assessment Center soll im Dreierteam Pascal Horat, Gerome Kamga und Gökhan Kaya abgewickelt werden. Die drei Teammitglieder kannten sich im Vornherein nicht. Pascal Horat und Gökhan Kaya studieren Elektrotechnik, Gerome Kamga Maschinenbau und ist ein Austauschstudent aus Deutschland/ Kamerun.

\subsection{Problem}
Die Problemstellung besteht darin, die wichtigsten Kompetenzen von Elektroingenieuren zu ermitteln and danach ein Verfahren (= Assessment) zu entwickeln, mit dessen Hilfe das Vorhandensein der gefundenen Kernkompetenzen bei Personen geprüft werden kann.
\subsection{Rahmenbedingungen}
Für die Bearbeitung des Projekts sind die vierzehn Wochen des Frühjahrssemesters 2017 vorgesehen, wobei alle Teammitglieder vollzeitlich Vorlesungen und Praktikas besuchen. Wöchentlich findet ein vierstündiger TKI-Block (= Teamkommunikation für Ingenieure) statt. Wobei die erste Hälfte immer aus einem Theorieblock des Dozenten besteht. Somit kann gesagt werden, dass uns von der Schule aus pro Woche zwei Lektionen für die Bearbeitung des Projekts zustehen. Hier muss aber angemerkt werden, dass die Projektarbeit nicht die einzige abzugebende Arbeit für dieses Modul ist. Jeder Student hat nämlich noch drei persönliche Lernbilanzen und drei Teamreviews (werden zusammen in den Teams verfasst) abzugeben. 
\subsection{Interessensvertreter/Stakeholder}
\subsubsection{Definition}
Als Stakeholder wird eine Person oder Gruppe bezeichnet, die ein berechtigtes Interesse am Verlauf oder Ergebnis eines Prozesses oder Projektes hat. QUELLE WIKIPEDIA

\subsubsection{Stakeholder unseres Projekts}
Da wir in unserem Projekt keine Firma als Projektpartner haben, ist neben unserem Team der einzige Stakeholder der Dozent. 

\section{Projektziele}
\subsubsection{Definition}
Projektziele sind die Aufstellung von möglichst quantifizierten Anforderungen, die erfüllt sein müssen, damit ein Projekt als erfolgreich abgeschlossen betrachtet werden kann.
\subsection{Übergeordnete Ziele, Projektgesamtziel}

\subsection{Systemziele}
\subsubsection{Definition}
Systemziele können erst gemessen werden, nachdem das Produkt/System für eine gewisse (definiert) Zeit im Einsatz gestanden hat. Die Verantwortung für die Erreichung der Systemziele trägt der Auftraggeber und nicht der Projektleiter.

Der Projektleiter muss selbstverständlich dafür besorgt sein, dass die notwendigen Ansprüche (= Anforderungen) an das Projekt erfüllt werden; d.h. dass das Produkt entsprechend ausgelegt ist.

\subsection{Abwicklungsziele}
\subsubsection{Definition}
Die Abwicklungsziele beziehen sich auf die Projektabwicklung und bezeichnen dort grundsätzlich die Inhalte der Meilensteine (Leistung, Termine, Kosten und Qualität) bzw. welche Arbeitsergebnisse wann in welcher Form vorliegen sollen (und wieviel das kosten darf).

\subsection{Abgrenzung}


\section{Risiken und Chancen}

\subsection{Risiken}

\subsection{Chancen}

\subsection{Kritische Erfolgsfaktoren}


\section{Problemlösung}

\subsection{Lösungsvarianten}

\subsection{Lösungsauswahl}


\section{Grobplanung}

\subsection{Rahmen}

\begin{tabular}{ | l | l | l | l |}
    \hline
    Projektbeginn: & blabla & Projektende: & balbla \\ \hline
\end{tabular}

\subsection{Meilensteine}

\begin{tabular}{ | l | l | l |}
    \hline
    Was? & Wer? & Ca. wann? \\ \hline
    blbal & blbal & blbal \\ \hline
    blbal & blbal & blbal \\ \hline
    blbal & blbal & blbal \\ \hline
\end{tabular}

\section{Ressourcen/Kosten}

\section{Projektorganisation}

\section{Machbarkeit}

\section{Unterschriften}