\chapter{Analyse Projektauftrag}
Die nachfolgende Analyse des Projektauftrags wurde anhand der Vorlage \textit{HSR\_Projektauftrag.docx} vorgenommen. Einige Punkte mögen daher eventuell unpassend erscheinen, wurden aber so gut wie möglich auf unser Projekt angewandt.
\section*{Assessment Center}

\section{Ausgangssituation und Kontext}
Der Stoff des Moduls Teamkommunikation für Ingenieure soll nicht nur theoretisch vermittelt, sondern auch praktisch erfahren werden können. Dazu wird ein industrienahes Projekt als Turngerät verwendet. Anhand von diesem Projekt sollen möglichst viele praktische Erfahrungen zum Thema Teamkommunikation gemacht werden. Das Turngerät ist in unserem Falle der Auftrag Assessment Center.

\subsection{Ist Situation}
Das Projekt Assessment Center soll im Dreierteam Pascal Horat, Gerome Kamga und Gökhan Kaya abgewickelt werden. Die drei Teammitglieder kannten sich im Vornherein nicht. Pascal Horat und Gökhan Kaya studieren Elektrotechnik, Gerome Kamga Maschinenbau und ist ein Austauschstudent aus Deutschland/ Kamerun.

\subsection{Problem}
Die Problemstellung besteht darin, die wichtigsten Kompetenzen von Elektroingenieuren zu ermitteln and danach ein Verfahren (= Assessment) zu entwickeln, mit dessen Hilfe das Vorhandensein der gefundenen Kernkompetenzen bei Personen geprüft werden kann.
\subsection{Rahmenbedingungen}
Für die Bearbeitung des Projekts sind die vierzehn Wochen des Frühjahrssemesters 2017 vorgesehen, wobei alle Teammitglieder vollzeitlich Vorlesungen und Praktikas besuchen. Wöchentlich findet ein vierstündiger TKI-Block (= Teamkommunikation für Ingenieure) statt. Wobei die erste Hälfte immer aus einem Theorieblock des Dozenten besteht. Somit kann gesagt werden, dass uns von der Schule aus pro Woche zwei Lektionen für die Bearbeitung des Projekts zustehen. Hier muss aber angemerkt werden, dass die Projektarbeit nicht die einzige abzugebende Arbeit für dieses Modul ist. Jeder Student hat nämlich noch drei persönliche Lernbilanzen und drei Teamreviews (werden zusammen in den Teams verfasst) abzugeben. 
\subsection{Interessensvertreter/Stakeholder}
\subsubsection{Definition}
Als Stakeholder wird eine Person oder Gruppe bezeichnet, die ein berechtigtes Interesse am Verlauf oder Ergebnis eines Prozesses oder Projektes hat. QUELLE WIKIPEDIA

\subsubsection{Stakeholder unseres Projekts}
Da wir in unserem Projekt keine Firma als Projektpartner haben, ist neben unserem Team der einzige Stakeholder der Dozent. 

\section{Projektziele}
\subsubsection{Definition}
Projektziele sind die Aufstellung von möglichst quantifizierten Anforderungen, die erfüllt sein müssen, damit ein Projekt als erfolgreich abgeschlossen betrachtet werden kann. QUELLE WIKIPEDIA

\subsubsection{Unsere Projektziele}
\begin{itemize}
\item Erfolgreich drei Kernkompetenzen ermitteln zu können
\item Zwei sinnvolle Übungen zu den Kernkompetenzen entwickeln
\item Die Übungen mit mindestens zwei Probanden durchzuführen, um schauen zu können ob man damit das Vorhandensein der Kernkompetenzen überprüfen kann
\item Den Projektbericht in der geforderten Qualität und Quantität zeitlich abgeben zu können
\item Möglichst viele Teamprozesse am eigenen Leib erfahren zu können
\end{itemize} 

\subsection{Systemziele}
\subsubsection{Definition}
Systemziele können erst gemessen werden, nachdem das Produkt/System für eine gewisse (definierte) Zeit im Einsatz gestanden hat. Die Verantwortung für die Erreichung der Systemziele trägt der Auftraggeber und nicht der Projektleiter.\\
Der Projektleiter muss selbstverständlich dafür besorgt sein, dass die notwendigen Ansprüche (= Anforderungen) an das Projekt erfüllt werden; d.h. dass das Produkt entsprechend ausgelegt ist.
QUELLE : anforderungsmanagement.ch

\subsubsection{Unsere Systemziele}
Da unser Assessment nicht wirklich eine Anwendung findet, können auch keine Systemziele definiert werden. 
\subsection{Abwicklungsziele}
\subsubsection{Definition}
Die Abwicklungsziele beziehen sich auf die Projektabwicklung und bezeichnen dort grundsätzlich die Inhalte der Meilensteine (Leistung, Termine, Kosten und Qualität) bzw. welche Arbeitsergebnisse wann in welcher Form vorliegen sollen (und wie viel das kosten darf). QUELLE : anforderungsmanagement.ch
\subsubsection{Unsere Abwicklungsziele}
Die definierten Meilensteine mit den entsprechenden Daten finden sich im Kapitel \ref{sec:Vorgehen} auf Seite \pageref{sec:Vorgehen}. 
\subsection{Abgrenzung}
Unser Projekt ist vollständig von den anderen Projekten und Teams im Modul abgegrenzt. Es gibt somit keine Überschneidungen. Selbstverständlich dürfen aber die andern Teams bei Fragen kontaktiert werden.
\section{Risiken und Chancen}
\subsection{Risiken}
\begin{itemize}
\item Ein Mitglied unseres Teams empfindet den Aufwand als zu hoch und meldet sich vollständig vom Modul ab
\item Die Teammitglieder finden bei der Bearbeitung des Projektes keinen gemeinsamen Nenner. Diese Möglichkeit besteht vor allem, weil sich die Mitglieder noch nicht gekannt haben im Vornherein
\item Die Zeitverhältnisse mit all den anderen Modulen sind zu knapp, es kann nicht die Zeit aufgewendet werden um eine qualitativ gute Arbeit abzuliefern
\end{itemize}
\subsection{Chancen}
\begin{itemize}
\item Die Teammitglieder empfinden das Projekt als eine positive Erfahrung und erleben Situationen, die ihnen im Berufsalltag helfen werden
\item Die eigene Erstellung eines Assessments fördert das Verständnis dafür, was von Firmen bei einem Bewerbungsverfahren verlangt wird
\end{itemize}
\subsection{Kritische Erfolgsfaktoren}
\begin{itemize}
\item Die Teammitglieder können genug Zeit in das Projekt investieren
\item Die Beziehungen innerhalb des Teams sind gut genug um eine qualitativ gute Arbeit abzuliefern
\end{itemize}

\section{Grobplanung}

\subsection{Zeitliche Abgrenzung}\label{Abgrenzung}
\begin{tabular}{ | l | l | l | l |}
    \hline
    Projektbeginn: & 20.02.2017 & Projektende: & 29.05.2017  \\ \hline
\end{tabular}

\subsection{Meilensteine}
Weitere Informationen dazu im Kapitel \ref{sec:Vorgehen} ab Seite \pageref{sec:Vorgehen}.\\

\begin{tabular}{ | p{0.3cm} | p{7.7cm} | p{4.3cm} | p{1.3cm} |}
    \hline
    \textbf{\#} & \textbf{Was?} & \textbf{Wer?} PaHo = Pascal Horat, GeKa = Gerome Kamga, GöKa = Gökhan Kaya & \textbf{Circa bis wann?} \\ \hline
    1 & Erstellen der Grobplanung in MS Project & PaHo & 12.04.17 \\ \hline
    2 & Erstellen der Struktur des Projektberichts in LaTex & GöKa & 12.04.17 \\ \hline
    3 & Fragekatalog ausarbeiten für Kernkompetenzen & GöKa, GeKa & 13.04.17 \\ \hline
    4 & Interviewpartner festlegen & PaHo, GeKa, GöKa & 21.04.17 \\ \hline
    5 & Kernkompetenzen von Ingenieuren ermitteln & PaHo, GeKa, GöKa & 28.04.17 \\ \hline
    6 & Die wichtigsten Kernkompetenzen eruieren & GeKa, GöKa & 01.05.17 \\ \hline
    7 & Zwei Übungen für Kernkompetenzen entwickeln & PaHo, GeKa, GöKa & 14.05.17 \\ \hline
    8 & Probanden organisieren & PaHo, GeKa, GöKa & 12.05.17 \\ \hline
    9 & Assessment mit Probanden durchführen & PaHo, GöKa & 19.05.17 \\ \hline
    10 & Assessment auswerten & PaHo, GöKa & 22.05.17 \\ \hline
    11 & Reflexion des Projekts & PaHo, GeKa, GöKa & 24.05.17 \\ \hline
    12 & Endüberarbeitung des Projektberichts & PaHo, GeKa, GöKa & 25.05.17 \\ \hline 
\end{tabular} 

\subsection{Diskrepanz Beginn Projekt}
Die zeitliche Diskrepanz zwischen dem Projektbeginn in Abschnitt \ref{Abgrenzung} und dem Beendigungsdatum des ersten Meilenstienes rührt daher, dass gerade am Anfang des Moduls sehr viel Zeit für die Erstellung der Teamreviews und der Lernbilanzen aufgewendet wurde.

\section{Ressourcen/Kosten}
\subsection{Personelle Ressourcen}
\begin{tabular}{| p{3cm} | p{4cm} | p{4cm} |}
\hline
\textbf{Arbeitskraft} & \textbf{Funktion} & \textbf{Stundenansatz} \\ \hline
Pascal Horat & Teamkoordinator & 45 CHF \\ \hline
Gerome Kamga & Teammitglied & 40 CHF \\ \hline
Gökhan Kaya & Teammitglied & 40 CHF \\ \hline

\end{tabular}\\

Als beratende und unterstützende Person trat der betreuende Dozent des Moduls, Dr. rer. pol Bruno Sternath, auf.

\subsection{Kosten}
\subsubsection{Personelle Kosten}
\subsubsection{Materielle Kosten}

\section{Projektorganisation}

\section{Machbarkeit}
Da schon zahlreiche andere Gruppen mit einer ähnlichen Vorbildung und Personalstärke dieses Projekt erfolgreich abschliessen konnten, wird die Machbarkeit als sehr gut eingeschätzt.