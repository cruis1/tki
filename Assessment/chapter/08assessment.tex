%08chapterKonsequenzen.tex
\chapter{Assessment}

\section{Auswahl der wichtigsten Kernkompetenzen}

Gemäss Bild \ref{fig:auswerkomp} haben sich folgende drei Kernkompetenzen als die Wichtigsten herausgestellt:

\begin{enumerate} 
\item{Logisches und analytisches Denken}
\item{Lernbereitschaft und Lernfähigkeit}
\item{Selbstmanagement und Selbstorganisation}
\end{enumerate}

\section{Übung 1: Logisches und analytisches Denken, Lernbereitschaft und Lernfähigkeit}

\section{Übung 2: Selbstmanagement und Selbstorganisation}

Mit Hilfe von dieser Übung soll ersichtlich werden, wie ausgeprägt die Kernkompetenz der Selbstorganisation beim Bewerber ist. Auch werden hier Elemente von Fachwissen und Lernbereitschaft angeschnitten. Anhand vordefinierter Kriterien soll es den Personen, welche das Assessment durchführen, möglich sein, eine valide und objektive Bewertung vornehmen zu können.

\subsection{Idee/ Grobbeschreibung}

Der Bewerber erhält vier bis fünf verschiedene, einfach scheinende Aufgaben welche er zu erledigen hat. Dies kann zum Beispiel das Ausrechnen von Schaltungsparametern einer Operationsverstärker-Schaltung, das Berechnen einer
mathematischen Aufgabe, das typografische korrigieren eines Messberichtes, das Antworten auf eine E-Mail, das Berechnen einer Emitterschaltung und so weiter sein. Die verschiedenen Aufgaben müssen zu unterschiedlichen Zeiten abgegeben werden, zusätzlich haben sie unterschiedliche Prioritäten. Die Abgabezeiten werden am Anfang mündlich bekannt gegeben. Der Bewerber hat Papier und Stift zur Verfügung.
Beim Erledigen der Aufgaben bemerkt er, dass die Reihenfolge der Aufgaben eine Rolle spielt, denn gewisse Aufgaben hängen von anderen ab. Die ganze Aufgabenstellung muss so ausgearbeitet sein, dass er nur mit
guter Planung (Zeitplanung / Prioritätenplanung), die Aufgaben zufriedenstellend erledigen kann.

Als Überschneidung mit der Lernbereitschaftsübung wird dem Bewerber zuallererst das Eisenhower-Prinzip erklärt, um dann direkt in oben beschriebener Übung zu schauen ob er es Anwenden kann, also bereit war, es zu Erlernen. 

Auf was von Assessmentseite geachtet wird:
\begin{itemize}
\item Macht er sich bei der Erläuterung der Aufgaben Notizen?
\item Schafft er sich eine Übersicht über die zu Erledigenden Arbeiten oder arbeitet er wild drauflos?
\item Erstellt er eine Zeitplanung?
\item Kategorisiert er die Aufgaben nach Dringlichkeit und Wichtigkeit (Eisenhower)?
\item Notiert er sich Fragen um Unklarheiten zu beseitigen (ihm muss vorher kommuniziert werden, das Fragen stellen erlaubt ist)?
\item Informiert er die Personen welche das Assessment durchführen wenn er es nicht schafft einen Auftrag innerhalb der Zeitfrist zu erledigen?
\end{itemize}

Mit dieser Übung wird eine Situation simuliert, welche in der Arbeitswelt so eins zu eins auftreten kann. Nämlich, verschiedene Aufgaben mit unterschiedlicher Priorität in einem begrenzten Zeitfenster erfolgreich bewältigen zu können.

\subsection{Detailbeschreibung}

An die vom Bewerber zu erledigenden Aufgaben werden folgende Kriterien gestellt:

\begin{itemize}
\item Sie soll einen Bezug zu Arbeiten haben, welche im Alltag eines Elektroingenieurs auftreten
\item Der Schwierigkeitsgrad soll so gewählt werden, dass sich der Bewerber nicht in der Aufgabe verlieren kann \item Es soll nur wenig Fachwissen zum Lösen der Aufgabe nötig sein, da das Überprüfen ebendieser nicht das Ziel ist
\item Es muss die Möglichkeit bestehen, die Aufgabe von anderen abhängig zu machen
\end{itemize}

Um testen zu können, ob der Bewerber sich zuerst ein Bild über alle zu erledigenden Aufgaben macht, muss jede Aufgabe von einer anderen abhängig sein, so dass es schlussendlich nur eine logische Abfolge gibt. Schafft er sich nämlich am Anfang keine Übersicht, sondern beginnt wahllos, so muss er rückwirkend Änderungen an vorhergehenden Aufgaben vornehmen, was ihn Zeit kostet.

Die einzig richtige Abfolge der Aufgaben ist folgende:

\subsubsection{Aufgabe 1: Korrektur Messbericht}
Dem Bewerber wird ein unvollständiger Messbericht mit typographischen Fehlern ausgehändigt, welche er korrigieren soll. % \ref{sec:Anhang}
Dies ist die erste Aufgabe, da die Resultate der Messung im Bericht Auswirkungen auf die Auslegung der Operationsverstärkerschaltung haben.

\subsubsection{Aufgabe 2: Berechnung Operationsverstärkerschaltung}
Das Resultat dieser Aufgabe ist eine vollständig berechnete Operationsverstärkerschaltung. Die Art der Schaltung wie auch die Werte der Bauelemente muss der Bewerber selber erarbeiten. Eine Notiz im Messbericht (vorherige Aufgabe) weist ihn darauf hin, ein Bauteil anders einzusetzen. Ignoriert er diese Notiz, muss er die ganze Berechnung wiederholen.

\subsubsection{Aufgabe 3: Beantwortung formelles E-Mail}
Ein Vorgesetzter braucht einige Angaben unseres Bewerbers, eine davon sind die Werte der Operationsverstärkerschaltung. Diese Aufgabe ist somit von der Vorhergehenden abhängig. Im E-Mail beschreibt der Vorgesetzte auch gleich noch den Typ des Transistors für die Emitterschaltung.

\subsubsection{Aufgabe 4: Berechnung Emitterschaltung}
Eine Emitterstufe soll berechnet werden. Diese sollte der Einfachheit halber die gleiche Ausgangsimpedanz wie die Operationsverstärkerschaltung aufweisen, ist somit also von Aufgabe 2 und drei abhängig. 

\subsubsection{Aufgabe ohne Reihenfolge: Kreuzworträtsel}
Der Teamleiter löst für sein Leben gerne Kreuzworträtsel in der Mittagspause. Da er ein sehr gründlicher Mensch ist, ist es ihm ein Bedürfnis, das Kreuzworträtsel vollständig zu haben. Leider kennt er nicht alle Antworten. Weil er aber weiss, dass der Bewerber ein ausgeprägtes Allgemeinwissen hat, gibt er ihm den Auftrag dieses während der Arbeitszeit zu vervollständigen. Diese Aufgabe hat die geringste Priorität, sie ist von keiner anderen Aufgabe abhängig und von ihr sind keine anderen Aufgaben abhängig.






\section{Ausarbeitung}

\section{Ablauf des Assessments}

\section{Aufgabenstellung}