%08chapterKonsequenzen.tex
\chapter{Assessment}\label{Assessment}

\section{Auswahl der wichtigsten Kernkompetenzen}

Gemäss Bild \ref{fig:auswerkomp} haben sich folgende drei Kernkompetenzen als die Wichtigsten herausgestellt:

\begin{enumerate} 
\item{Analytisches und systematisches Denken}
\item{Lernbereitschaft und Lernfähigkeit}
\item{Selbstmanagement und Selbstorganisation}
\end{enumerate}

\section{Übung 1: Logisches und analytisches Denken, Lernbereitschaft und Lernfähigkeit}

In der ersten Übung geht es vor allem darum, die Lernbereitschaft und die Lernfähigkeit zu testen. Als Nebeneffekte werden zusätzlich logisches und analytisches Denken, sowie etwas Selbstmanagement und Selbstorganisation geprüft.

\subsection{Idee}

Die Lernfähigkeit und Lernbereitschaft wird gemäss Wikipedia \cite{wiki:Lernfahigkeit} wie folgt definiert:
\begin{quote} 
\textit{Unter Lernfähigkeit wird die Bereitschaft und Fähigkeit verstanden, Ausbildungsinhalte eigenständig, langfristig aufzunehmen, logisch zu ordnen, zu verarbeiten und aus eigenen Fehlern zu lernen.}
\end{quote}

Um die Lernbereitschaft und Lernfähigkeit zu prüfen, wird dem Prüfling ein Thema vorgesetzt, womit er sich nicht oder so wenig wie möglich auskennt, damit die Probanden keine Vorteile gegenüber anderen geniessen können. Anschliessend ist es die Aufgabe der geprüften Personen, sich mit dem neuen Themengebiet zu befassen und schliesslich kurz zu präsentieren.

Wichtig an dieser Stelle ist das Thema selbst. Hier werden komplexe Themen bevorzugt. Voraussetzungen sind, dass logisches und analytisches Denken sowie Selbstmanagement und Selbstorganisation gefordert sind, um sich dem neuen Themengebiet möglichst umfassend zu nähern. Das Thema selbst ist technischer Natur, hat aber nichts mit dem zu tun, was die Studenten kennen oder irgendeinen Bezug dazu haben. Durch das vorkommen von vielen Fachwörtern ist es nötig die Informationsbeschaffung gut zu organisieren (Selbstmanagement). Das Thema ist komplex genug, damit logisches und analytisches Denken nicht zu vernachlässigen sind.

Das Ergebnis des Tests ist ein Zusammenspiel aus all diesen und vielen weiteren Kompetenzen. Aus diesem Grunde ist es sehr schwierig, diese anhand der Ergebnisse und Beobachtungen zu trennen. Hier wird die Trennung soweit wie möglich angestrebt.

\subsection{Detailbeschreibung}

Das Thema, das unsere Kriterien gut erfüllt, ist eines aus der Biologie, nämlich die Polymerase Kettenreaktion (PCR). Wir gehen davon aus, dass keines der Studenten ein Vorwissen dazu hat. Es ist ein umfassender Wikipedia-Artikel verfügbar, der sehr viele Fachbegriffe enthält, die ein HSR Student im technischen Bereich nicht kennen sollte. Nur Wikipedia allein wird bei den Probanden sehr wahrscheinlich viele Fragezeichen hinterlassen, was wir bei der Präsentation merken sollten. 

\subsubsection{Prüfungsfrage}
Ein Blatt mit dem folgenden Text wird dem Studenten hingelegt:

\textit{Sie haben 15min Zeit, um sich in ein vorgegebenes Themengebiet einzulesen. \\\vspace*{2mm}
Das Thema ist PCR (Polymerase Kettenreaktion). \\
Anschliessend sollen Sie dieses Thema so ausführliche wie möglich Präsentieren. Sie dürfen alle Hilfsmittel verwenden inkl. Internet. Für die Präsentation sind nur Ihre selbst verfassten Notizen zugelassen.}
\subsection{Bewertung}

\textbf{Während dem Test:}
\begin{enumerate}
\item Verschafft er sich zuerst einen Überblick?
\item Ist eine Methodik zu erkennen oder wird wild hin und her gesucht?
\end{enumerate}

\textbf{Nach dem Test:}
\begin{enumerate}
\item Wie sicher fühlt sich die Person während der Präsentation?
\item Wie genau/verständlich konnte der Prozess erklärt werden. Wurden die wichtigsten Punkte erwähnt?
\item Wurden Fremdwörter verwendet oder alles in eigene Worte übersetzt?
\item Wurde visuell gearbeitet?
\item Wurden die Erklärungen strukturiert?
\item Wie viel wurde über den Prozess gesprochen? Wurde unnötig über Nebensächlichkeiten geredet?
\item Wirkte die Person interessiert?
\end{enumerate}
\vspace{3mm}

\textit{Legende}

lb \& lf = Lernbereitschaft und Lernfähigkeit\\
lg \& an = logisches und analytisches Denken \\
sm \& so = Selbstmanagement und Selbstorganisation

\subsubsection{Verschafft er sich zuerst einen Überblick?}
\begin{tabular}{| l | c | c | c |}
  \hline	
  \textbf{Ausprägung} & \textbf{Punkte lb \& lf} & \textbf{Punkte lg \& an} & \textbf{Punkte sm \& so} \\
  \hline  		
  Schafft sich keinen Überblick & 0  & 0 & 0 \\ 
  \hline
  Schafft sich fast keinen Überblick & 1 & 0 & 1 \\ 
  \hline
  Schafft sich Überblick & 2 & 0 & 2 \\
  \hline  
  Studiert die Aufgabenstellung Gründlich & 3 & 0 &  3 \\
  \hline  
\end{tabular}
  
\subsubsection{Ist eine Methodik zu erkennen oder wird wild ziellos gesucht?}
\begin{tabular}{| l | c | c | c |}
  \hline	
  \textbf{Ausprägung} & \textbf{Punkte lb \& lf} & \textbf{Punkte lg \& an} & \textbf{Punkte sm \& so} \\
  \hline  		
  Gar keine Methodik & 0  & 0 & 0 \\ 
  \hline
  Nur wenig Methodik & 1 & 0 & 1 \\ 
  \hline
  Gute Methodik & 2 & 0 & 2 \\
  \hline  
  Sehr strukturierte Vorgehensweise & 3 & 0 &  3 \\
  \hline  
\end{tabular}

\subsubsection{Wie sicher fühlt sich die Person während der Präsentation?}
\begin{tabular}{| l | c | c | c |}
  \hline	
  \textbf{Ausprägung} & \textbf{Punkte lb \& lf} & \textbf{Punkte lg \& an} & \textbf{Punkte sm \& so} \\
  \hline  		
  Sehr unsicher & 0  & 0 & 0 \\ 
  \hline
  Eher unsicher & 0 & 0 & 0 \\ 
  \hline
  Eher sicher & 0 & 1 & 1 \\
  \hline  
  Sehr selbstsicher & 0 & 2 &  2 \\
  \hline  
\end{tabular}

\subsubsection{Wie genau/verständlich konnte der Prozess erklärt werden. Wurden die wichtigsten Punkte erwähnt?}
\begin{tabular}{| l | c | c | c |}
  \hline	
  \textbf{Ausprägung} & \textbf{Punkte lb \& lf} & \textbf{Punkte lg \& an} & \textbf{Punkte sm \& so} \\
  \hline  		
  Hat Prozess nicht verstanden & 0  & 0 & 0 \\ 
  \hline
  Hat Prozess etwas verstanden & 1 & 1 & 1 \\ 
  \hline
  Hat Prozess gut verstranden & 2 & 2 & 2 \\
  \hline  
  Hat Prozess hervorragend verstanden & 3 & 3 &  3 \\
  \hline  
\end{tabular}

\subsubsection{Wurden Fremdwörter verwendet oder alles in eigene Worte übersetzt?}
\begin{tabular}{| l | c | c | c |}
  \hline	
  \textbf{Ausprägung} & \textbf{Punkte lb \& lf} & \textbf{Punkte lg \& an} & \textbf{Punkte sm \& so} \\
  \hline  		
  Keine Fremdwörter verwendet & 0  & 0 & 0 \\ 
  \hline
  Wenig Fremdwörter verwendet & 1 & 0 & 0 \\ 
  \hline
  Einige Fremdwörter verwendet & 2 & 0 & 0 \\
  \hline  
  Sehe viele Fremdwörter verwendet & 3 & 0 &  0 \\
  \hline  
\end{tabular}

\subsubsection{Wurde visuell gearbeitet? (Damit sind die Notizen inbegriffen)}
\begin{tabular}{| l | c | c | c |}
  \hline	
  \textbf{Ausprägung} & \textbf{Punkte lb \& lf} & \textbf{Punkte lg \& an} & \textbf{Punkte sm \& so} \\
  \hline  		
  Keine & 0  & 0 & 0 \\ 
  \hline
  Kaum & 1 & 1 & 1 \\ 
  \hline
  Einige & 2 & 2 & 2 \\
  \hline  
  Sehe viele & 3 & 3 & 3 \\
  \hline  
\end{tabular}

\subsubsection{Wurden die Erklärungen strukturiert?}
\begin{tabular}{| l | c | c | c |}
  \hline	
  \textbf{Ausprägung} & \textbf{Punkte lb \& lf} & \textbf{Punkte lg \& an} & \textbf{Punkte sm \& so} \\
  \hline  		
  Gar nich & 0  & 0 & 0 \\ 
  \hline
  Kaum & 1 & 1 & 1 \\ 
  \hline
  Genügend & 2 & 2 & 2 \\
  \hline  
  Sehr gute Struktur & 3 & 3 & 3 \\
  \hline  
\end{tabular}

\subsubsection{Wie viel wurde über den Prozess gesprochen? Wurde unnötig über Nebensächlichkeiten geredet?}
\begin{tabular}{| l | c | c | c |}
  \hline	
  \textbf{Ausprägung} & \textbf{Punkte lb \& lf} & \textbf{Punkte lg \& an} & \textbf{Punkte sm \& so} \\
  \hline  		
  Viele Nebensächlichkeiten & 0  & 0 & 0 \\ 
  \hline
  Einige Nebensächlichkeiten & 1 & 1 & 1 \\ 
  \hline
  Kaum Nebensächlichkeiten & 2 & 2 & 2 \\
  \hline  
  Keine Nebensächlichkeiten & 3 & 3 & 3 \\
  \hline  
\end{tabular}

\subsubsection{Wirkte die Person interessiert?}
\begin{tabular}{| l | c | c | c |}
  \hline	
  \textbf{Ausprägung} & \textbf{Punkte lb \& lf} & \textbf{Punkte lg \& an} & \textbf{Punkte sm \& so} \\
  \hline  		
  Gar nicht & 0  & 0 & 0 \\ 
  \hline
  Etwas & 1 & 0 & 0 \\ 
  \hline
  Interessiert & 2 & 0 & 0 \\
  \hline  
  Sehr Interessiert & 3 & 0 & 0 \\
  \hline  
\end{tabular}



\section{Übung 2: Selbstmanagement und Selbstorganisation}

Mit Hilfe von dieser Übung soll ersichtlich werden, wie ausgeprägt die Kernkompetenz der Selbstorganisation beim Bewerber ist. Auch werden hier Elemente von systematischen Denkens und Lernbereitschaft angeschnitten. Anhand vordefinierter Kriterien soll es den Personen, welche das Assessment durchführen, möglich sein, eine valide und objektive Bewertung vornehmen zu können.

\subsection{Idee/ Grobbeschreibung}

Der Bewerber erhält vier bis fünf verschiedene, einfach scheinende Aufgaben welche er zu erledigen hat. Dies kann zum Beispiel das Ausrechnen von Schaltungsparametern einer Operationsverstärker-Schaltung, das Berechnen einer
mathematischen Aufgabe, das typografische korrigieren eines Messberichtes, das Antworten auf eine E-Mail, das Berechnen einer Emitterschaltung und so weiter sein. Die verschiedenen Aufgaben müssen zu unterschiedlichen Zeiten abgegeben werden, zusätzlich haben sie unterschiedliche Prioritäten. Die Abgabezeiten werden am Anfang mündlich bekannt gegeben. Der Bewerber hat Papier und Stift zur Verfügung.
Beim Erledigen der Aufgaben bemerkt er, dass die Reihenfolge der Aufgaben eine Rolle spielt, denn gewisse Aufgaben hängen von anderen ab. Die ganze Aufgabenstellung muss so ausgearbeitet sein, dass er nur mit
guter Planung (Zeitplanung / Prioritätenplanung), die Aufgaben zufriedenstellend erledigen kann.

Als Überschneidung mit der Lernbereitschaftsübung wird dem Bewerber zuallererst das Eisenhower-Prinzip erklärt, um dann direkt in oben beschriebener Übung zu schauen ob er es Anwenden kann, also bereit war, es zu Erlernen. 

Auf was von Assessmentseite geachtet wird:
\begin{itemize}
\item Macht er sich bei der Erläuterung der Aufgaben Notizen?
\item Schafft er sich eine Übersicht über die zu Erledigenden Arbeiten oder arbeitet er wild drauflos?
\item Erstellt er eine Zeitplanung?
\item Kategorisiert er die Aufgaben nach Dringlichkeit und Wichtigkeit (Eisenhower)?
\item Notiert er sich Fragen um Unklarheiten zu beseitigen (ihm muss vorher kommuniziert werden, das Fragen stellen erlaubt ist)?
\item Informiert er die Personen welche das Assessment durchführen wenn er es nicht schafft einen Auftrag innerhalb der Zeitfrist zu erledigen?
\end{itemize}

Mit dieser Übung wird eine Situation simuliert, welche in der Arbeitswelt so eins zu eins auftreten kann. Nämlich, verschiedene Aufgaben mit unterschiedlicher Priorität in einem begrenzten Zeitfenster erfolgreich bewältigen zu können.

\subsection{Detailbeschreibung}

An die vom Bewerber zu erledigenden Aufgaben werden folgende Kriterien gestellt:

\begin{itemize}
\item Sie soll einen Bezug zu Arbeiten haben, welche im Alltag eines Elektroingenieurs auftreten
\item Der Schwierigkeitsgrad soll so gewählt werden, dass sich der Bewerber nicht in der Aufgabe verlieren kann \item Es soll nur wenig Fachwissen zum Lösen der Aufgabe nötig sein, da das Überprüfen ebendieser nicht das Ziel ist
\item Es muss die Möglichkeit bestehen, die Aufgabe von anderen abhängig zu machen
\end{itemize}

Um testen zu können, ob der Bewerber sich zuerst ein Bild über alle zu erledigenden Aufgaben macht, muss jede Aufgabe von einer anderen abhängig sein, so dass es schlussendlich nur eine logische Abfolge gibt. Schafft er sich nämlich am Anfang keine Übersicht, sondern beginnt wahllos, so muss er rückwirkend Änderungen an vorhergehenden Aufgaben vornehmen, was ihn Zeit kostet.

Die einzig richtige Abfolge der Aufgaben ist folgende:

\subsubsection{Aufgabe 1: Korrektur Messbericht}
Dem Bewerber wird ein unvollständiger Messbericht mit typographischen Fehlern ausgehändigt, welche er korrigieren soll. % \ref{sec:Anhang}
Dies ist die erste Aufgabe, da die Resultate der Messung im Bericht Auswirkungen auf die Auslegung der Operationsverstärkerschaltung haben.

\subsubsection{Aufgabe 2: Berechnung Operationsverstärkerschaltung}
Das Resultat dieser Aufgabe ist eine vollständig berechnete Operationsverstärkerschaltung. Der Wert einiger Bauelemente muss der Bewerber selber erarbeiten. Eine Notiz im Messbericht (vorherige Aufgabe) weist ihn darauf hin, ein Bauteil anders einzusetzen. Ignoriert er diese Notiz, muss er die ganze Berechnung wiederholen.

\subsubsection{Aufgabe 3: Beantwortung E-Mail}
Ein Vorgesetzter braucht einige Angaben unseres Bewerbers, eine davon sind die Werte der Operationsverstärkerschaltung. Diese Aufgabe ist somit von der Vorhergehenden abhängig. Im E-Mail beschreibt der Vorgesetzte auch gleich noch den Typ des Transistors für die Emitterschaltung.

\subsubsection{Aufgabe 4: Berechnung Emitterschaltung}
Eine Emitterstufe soll berechnet werden. Diese sollte der Einfachheit halber als RC den gleichen Widerstandswert aufweisen wie der Feedback-Widerstand der Operationsverstärkerschaltung, ist somit also von Aufgabe zwei und drei abhängig. 

\subsubsection{Aufgabe ohne Reihenfolge: Kreuzworträtsel}
Der Teamleiter löst für sein Leben gerne Kreuzworträtsel in der Mittagspause. Da er ein sehr gründlicher Mensch ist, ist es ihm ein Bedürfnis, das Kreuzworträtsel vollständig zu haben. Leider kennt er nicht alle Antworten. Weil er aber weiss, dass der Bewerber ein ausgeprägtes Allgemeinwissen hat, gibt er ihm den Auftrag dieses während der Arbeitszeit zu vervollständigen. Diese Aufgabe hat die geringste Priorität, sie ist von keiner anderen Aufgabe abhängig und von ihr sind keine anderen Aufgaben abhängig.

Die Abgabezeiten der einzelnen Aufgaben und deren Priorität sind wie folgt vorgegeben:
\begin{center}
  \begin{tabular}{ | p{7cm} | p{4cm} |}
   \hline
   \textbf{Aufgabe} & \textbf{Abgabezeitpunkt} \\ \hline
   Beantwortung Mail & nach 10 min \\ \hline
   Berechnung Operationsverstärkerschaltung & nach 10 min \\ \hline
   Korrektur Messbericht & nach 15 min \\ \hline
   Kreuzworträtsel & nach 20 min \\ \hline
   Emitterstufe & nach 20 min\\ \hline
  \end{tabular}
\end{center}


\begin{center}
  \begin{tabular}{ | p{7cm} | p{4cm} |}
   \hline
   \textbf{Aufgabe} & \textbf{Wichtigkeit/Priorität} \\ \hline
   Beantwortung Mail & 1 -> höchste Priorität\\ \hline
   Berechnung Operationsverstärkerschaltung & 2 \\ \hline
   Korrektur Messbericht & 1 \\ \hline
   Kreuzworträsel & 4 -> tiefste Priorität\\ \hline
   Emitterstufe & 3\\ \hline
  \end{tabular}
\end{center}


\subsection{Am Anfang dem Bewerber mitzuteilende Informationen}

Bevor der Bewerber mit der Lösung der Aufgaben beginnen kann, muss der Prüfer folgende Informationen an ihn weitergeben:

\begin{itemize}
\item Der Prüfer soll sagen, dass mit dieser Übung versucht wird eine Situation zu simulieren, welche in der Arbeitswelt so auftreten kann. Nämlich, verschiedene Aufgaben mit unterschiedlicher Priorität in einem begrenzten Zeitfenster erfolgreich bewältigen zu können.
\item Das Eisenhower-Prinzip erklären und verdeutlichen, dass es in dieser Übung ein nützliches Hilfsmittel wäre.
\item Kurz jede Aufgabe gemäss Aufgabenbeschreibung erläutern.
\item Abgabezeiten und Prioritäten mündlich mitteilen.
\item Dem Bewerber ist deutlich mitzuteilen, dass bei dieser Übung vor allem auf seine Selbstorganisation geachtet und nicht sein Fachwissen geprüft wird.  
\item Es muss dem Geprüften mitgeteilt werden, dass manche Aufgaben untereinander verknüpft sind.
\item Der Prüfer soll deutlich machen, dass er über den Arbeitsstand des Geprüften mündlich informiert werden möchte, besonders wenn eine zeitliche Abgabe einer Aufgabe nicht möglich ist.
\item Es muss mitgeteilt werden, dass Fragen nur beantwortet werden, wenn wieder volle fünf Minuten abgelaufen sind, somit also nur alle fünf Minuten.
\item Der Widerstandswert des Kollektor-Widerstandes von Schaltung T0455 soll gleich dem Feedback-Widerstand in T0454 sein.
\end{itemize}

\subsection{Bewertung}

Um eine objektive Bewertung vornehmen zu können, müssen die Bewertungskriterien und ihre Gewichtung im Vornherein klar definiert sein.

Die oben erwähnten Kriterien werden erweitert und mit folgender Gewichtung versehen:


\subsubsection{Macht der Bewerber sich Notizen?}
Der Ablauf der Übung wird dem Bewerber nur mündlich mitgeteilt. Ihm werden Stift und Papier bereitgestellt. Die mitgeteilten Informationen beinhalten auch die Prioritäten und Abgabezeiten der verschiedenen Aufgaben. Wenn der Bewerber sich keine Notizen macht, wird er sich wahrscheinlich nicht alles merken können. 

Bei diesem Punkt wird darauf geachtet ob und in welchem Masse der Bewerber Notizen nimmt.

\begin{center}
  \begin{tabular}{ | p{5cm} | p{1cm} |}
   \hline
   \textbf{Ausprägung} & \textbf{Punkte} \\ \hline
   macht keine Notizen & 0 \\ \hline
   macht wenige Notizen & 1 \\ \hline
   macht viele Notizen & 2 \\ \hline
   notiert sich alles  & 3\\ \hline
  \end{tabular}
\end{center}


\subsubsection{Schafft er sich eine Übersicht über die Arbeiten?}
Die Aufgaben haben verschiedene Prioritäten und sind auch voneinander abhängig. Darum ist es von grosser Wichtigkeit, sich einen Überblick zu verschaffen, bevor man mit den einzelnen Aufgaben beginnt.

Bei der Bewertung wird darauf geachtet, ob der Bewerber sich mit en Aufgaben auseinandersetzt oder einfach wahllos zu arbeiten beginnt.

\begin{center}
  \begin{tabular}{ | p{6.5cm} | p{1cm} |}
   \hline
   \textbf{Ausprägung} & \textbf{Punkte} \\ \hline
   schafft sich keine Übersicht & 0 \\ \hline
   schafft sich fast keine Übersicht & 1 \\ \hline
   schafft sich Übersicht & 2 \\ \hline
   studiert die Aufgabenstellungen gründlich  & 3\\ \hline
  \end{tabular}
\end{center}

\subsubsection{Erstellt er eine Zeitplanung?}
Da die Aufgabenstellungen verschiedene Abgabezeiten haben, ist es von Nöten, sich eine kurze Zeitplanung mit eventueller Reservezeit zu erstellen.


\begin{center}
  \begin{tabular}{ | p{7cm} | p{1cm} |}
   \hline
   \textbf{Ausprägung} & \textbf{Punkte} \\ \hline
   erstellt keine Zeitplanung & 0 \\ \hline
   erstellt so etwas wie eine Zeitplanung & 1 \\ \hline
   erstellt eine gute Zeitplanung & 2 \\ \hline
   erstellt detail. Zeitplanung inkl. Reservezeiten  & 3\\ \hline
  \end{tabular}
\end{center}

\subsubsection{Wendet er das vorgestellte Eisenhower-Prinzip an?}
Die Aufgaben sind je nach Abgabezeitpunkt dringender oder weniger dringend und je nach Priorität wichtiger oder weniger wichtig. Mit einem Eisenhower-Diagramm kann der Bewerber sich schnell einen Überblick über die verschiedenen Aufgaben schaffen.

\begin{center}
  \begin{tabular}{ | p{7cm} | p{1cm} |}
   \hline
   \textbf{Ausprägung} & \textbf{Punkte} \\ \hline
   wendet Eisenhower-Prinzip nicht an & 0 \\ \hline
   versucht das Eisenhower-Prinzip anzuwenden & 1 \\ \hline
   wendet Eisenhower-Prinzip richtig an & 2 \\ \hline
   wendet Eisenhower-Prinzip korrekt an und leitet dadurch Konsequenzen für die Bearbeitung der Aufgaben ab  & 3\\ \hline
  \end{tabular}
\end{center}

Die Punktzahl von diesem Punkt fliesst in die Berwertung der Kernkompetenzen der Lernbereitschaft und der Selbstorganistion ein.

\subsubsection{Notiert er sich Fragen?}
Für das Lösen der Aufgaben sind zum Teil weitere Informationen nötig. Darum ist es wichtig das der Bewerber sich Fragen notiert, sodass er diese dem Prüfer zu einem angekündigten Zeitpunkt stellen kann. Stellt er keine Fragen, ist es ihm nicht möglich alle Aufgaben korrekt zu lösen.


\begin{center}
  \begin{tabular}{ | p{7cm} | p{1cm} |}
   \hline
   \textbf{Ausprägung} & \textbf{Punkte} \\ \hline
   notiert keine Fragen & 0 \\ \hline
   notiert sich wenige relevante Fragen & 1 \\ \hline
   notiert einige relevante Fragen & 2 \\ \hline
   notiert und stellt alle relevanten Fragen  & 3\\ \hline
  \end{tabular}
\end{center}

\subsubsection{Informiert er über seinen Arbeitsstand?}
Für den Bewerber wird es zeitlich schwierig sein, alle Aufgaben erfolgreich zu lösen. Darum ist es für einen allfälligen Vorgesetzten wichtig, über den Arbeitsstand des Arbeiters informiert zu werden, so dass dieser die Planung anpassen kann. Dies beinhaltet zum Beispiel anzukündigen, wenn ein Auftrag nicht innerhalb der Frist erledigt werden kann.

\begin{center}
  \begin{tabular}{ | p{7cm} | p{1cm} |}
   \hline
   \textbf{Ausprägung} & \textbf{Punkte} \\ \hline
   informiert den Prüfer nicht über den Arbeitsstand & 0 \\ \hline
   gibt fast keine Informationen weiter & 1 \\ \hline
   gibt viele Informationen an den Prüfer weiter & 2 \\ \hline
   gibt alle relevanten Informationen zeitgerecht an den Prüfer weiter & 3\\ \hline
  \end{tabular}
\end{center}

\subsubsection{Elemente des analytischen und systematischen Denkens}
Da Elemente des analytischen und systematischen Denkens in dieser Übung auftreten, gibt es zwei weitere Bewertungsraster für diese beiden Punkte.

\begin{center}
  \begin{tabular}{ | p{11cm} | p{1cm} |}
   \hline
   \textbf{Ausprägung analytisches Denken} & \textbf{Punkte} \\ \hline
   es fällt dem Geprüften sehr schwer die Aufgabenstellungen zu verstehen & 0 \\ \hline
   der Geprüfte bei einigen Aufgabenstellungen Mühe sie zu verstehen  & 1 \\ \hline
   der Geprüfte hat wenig Mühe die Aufgabenstellungen zu verstehen & 2 \\ \hline
   dem Geprüften waren alle Aufgabenstellungen sofort klar & 3\\ \hline
  \end{tabular}
\end{center}

\begin{center}
  \begin{tabular}{ | p{11cm} | p{1cm} |}
   \hline
   \textbf{Ausprägung systematisches Denken} & \textbf{Punkte} \\ \hline
   der Geprüfte hat grosse Mühe, die Verknüpfungen zwischen den Aufgaben zu verstehen & 0 \\ \hline
    der Geprüfte hat teilweise Mühe, die Verknüpfungen zwischen den Aufgaben zu verstehen & 1 \\ \hline
   der Geprüfte findet und versteht die Abhängigkeiten der Aufgaben & 2 \\ \hline
   findet und versteht die Abhängigkeiten der Aufgaben auf Anhieb & 3\\ \hline
  \end{tabular}
\end{center}


\section{Ablauf des Assessments}

Um eine Aussage über die Anwendbarkeit und die Effektivität unseres Assessments machen zu können, haben wir uns entschieden, dieses mit drei Probanden durchzuführen. Die Anforderungen an diese Testpersonen sind gute Deutschkenntnisse und dass sie Elektrotechnik studieren, da das Assessment auf ebendiese Studienrichtung ausgerichtet ist. Nachdem wir drei Probanden ausfindig gemacht hatten und diese überzeugen konnten mitzumachen, war der Ablauf wie folgend:

\begin{enumerate}
\item Der Arbeitsplatz des Probanden wurde eingerichtet, dazu gehörte
\begin{itemize}
\item ein Computer für die Internetrecherche
\item Schreibzug und Notizpapier
\item die Aufgabenstellung für die Übung 1
\item die fünf Aufgabenstellungen für die Aufgabe 2
\item ein Videoaufnahmegerät um das Assessment aufzuzeichnen
\end{itemize}
\item Übung 1 wurde durch Gökhan Kaya erläutert und durchgeführt
\item Eine kurze Auswertung der Übung wurde mit dem Probanden durch den Prüfer durchgeführt
\item Übung 2 wurde durch Pascal Horat erläutert und durchgeführt
\item Eine kurze Auswertung der Übung wurde mit dem Probanden durch den Prüfer durchgeführt
\item Feedback vom Geprüften zum Verbessern der Übung wurde eingeholt
\end{enumerate}


Wie oben schon angedeutet wurde der Proband während dem ganzen Ablauf gefilmt. Dies einerseits um das Video während dem Auswerten des Assessments konsultieren zu können, andererseits um belegen zu können, dass das Assessment auch wirklich mit Probanden durchgeführt wurde.
