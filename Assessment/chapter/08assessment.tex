%08chapterKonsequenzen.tex
\chapter{Assessment}

\section{Auswahl der wichtigsten Kernpetenzen}

Gemäss Bild \ref{fig:auswerkomp} haben sich folgende drei Kernkompetenzen als die Wichtigsten herausgestellt:

\begin{enumerate} 
\item{Logisches und analytisches Denken}
\item{Lernbereitschaft und Teamfähigkeit}
\item{Selbstmanagement und Selbstorganisation}
\end{enumerate}

\section{Methoden für ein Tool}

\section{Beobachtungsinstrument Selbstmanagement und Selbstorganisation}

Anhand dieses Beobachtungsinstruments soll ersichtlich werden, wie ausgeprägt die Kernkompetenz der Selbstorganisation beim Bewerber ist. Anhand vordefinierter Kriterien soll es den Personen, welche das Assessment durchführen, möglich sein, eine valide und objektive Bewertung vornehmen zu können.

\subsection{Idee/ Grobbeschreibung}

Der Bewerber erhält drei bis vier verschiedene, einfach scheinende Aufgaben welche er zu erledigen hat. Dies kann zum Beispiel das Ausrechnen von Schaltungsparametern einer Operationsverstärker-Schaltung, das Berechnen einer
mathematischen Aufgabe, das typografische korrigieren eines Messberichtes, das Antworten auf eine E-Mail, das Berechnen einer Emitterschaltung und so weiter sein. Die verschiedenen Aufgaben müssen zu unterschiedlichen Zeiten abgegeben werden, zusätzlich haben sie unterschiedliche Prioritäten. Die Abgabezeiten werden am Anfang mündlich bekannt gegeben. Der Bewerber hat Papier und Stift zur Verfügung.
Beim Erledigen der Aufgaben bemerkt er, dass die Reihenfolge der Aufgaben eine Rolle spielt, denn gewisse Aufgaben hängen von anderen ab. Die ganze Aufgabenstellung muss so ausgearbeitet sein, dass er nur mit
guter Planung (Zeitplanung / Prioritätenplanung), die Aufgaben zufriedenstellend erledigen kann.


Auf was von Assessmentseite geachtet wird:
- Macht er sich bei der Erläuterung der Aufgaben Notizen?
- Schafft er sich eine Übersicht über die zu Erledigenden Arbeiten oder arbeitet er wild drauflos?
- Erstellt er eine Zeitplanung?
- Kategorisiert er die Aufgaben nach Dringlichkeit und Wichtigkeit (Eisenhower)?
- Notiert er sich Fragen um Unklarheiten zu beseitigen (ihm muss vorher kommuniziert werden, das Fragen stellen erlaubt ist)?
- Informiert er die Personen welche das Assessment durchführen wenn er es nicht schafft einen Auftrag innerhalb der Zeitfrist zu erledigen

Mit dieser Übung wird eine Situation simuliert, welche in der Arbeitswelt so eins zu eins auftreten kann. Nämlich, verschiedene Aufgaben mit unterschiedlicher Priorität in einem begrenzten Zeitfenster erfolgreich bewältigen zu können.

\subsection{Detailbeschreibung}

An die vom Bewerber zu erledigenden Aufgaben werden folgende Kriterien gestellt:

\begin{itemize}
\item Sie soll einen Bezug zu Arbeiten haben, welche im Alltag eines Elektroingenieurs auftreten
\item Der Schwierigkeitsgrad soll so gewählt werden, dass sich der Bewerber nicht in der Aufgabe verlieren kann \item Es soll nur wenig Fachwissen zum Lösen der Aufgabe nötig sein, da das Überprüfen ebendieser nicht das Ziel ist
\item Es muss die Möglichkeit bestehen, die Aufgabe von anderen abhängig zu machen
\end{itemize}

\subsubsection{Aufgabenbeschreibung}

Aufgabe 1...
Aufgabe 2...
Aufgabe 3...

(inkl. Elementen von Fachwissen und Lernbereitschaft)

Als Überschneidung mit der Lernbereitschaftsübung könnte man ihm zuerst das Eisenhower-Prinzip erklären, um dann direkt in oben beschriebener Übung zu schauen ob er es Anwenden kann, also bereit war, es zu erlernen. 

\section{Ausarbeitung}

\section{Ablauf des Assessments}

\section{Aufgabenstellung}