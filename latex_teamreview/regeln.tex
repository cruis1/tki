\documentclass[11pt]{article}

%% Text-Encoding festlegen. Mit utf8 oder utf8x funktionieren Umlaute wie gewohnt.
%% (mit Bibtex funktioniert nur utf8)
\usepackage[utf8x]{inputenc}

%% Sprachdatei für Trennregeln, Datum-Format und ähnliches festlegen
\usepackage[german]{babel}  % nötig für Umlaute
% \usepackage[english]{babel}

%% optimiert das typographische Erscheinungsbild
\usepackage{microtype}

%% erlaubt Listen einfacher zu formatieren (bietet nosep für kompakte Listen)
\usepackage{enumitem}
%% erlaubt hübsche Tabellen über mehrere Seiten, beinhaltet booktabs (\toprule, \midrule, ...)
\usepackage{ctable}
%% ermöglicht farbigen Text ({\color{red} ...})
\usepackage{xcolor} 

%% erweiterte Funktionalität für Formeln (Pakete der American Mathematical Society)
\usepackage{amsfonts,amsmath,amsthm,amssymb}

%% vordefinierte Einheiten, einfaches Angeben von Einheiten (\SI{8 \pm 1}{cm})
%%   die Unsicherheit soll mit +- abgetrennt werden
\usepackage[separate-uncertainty]{siunitx}
\sisetup{
    range-units = single,       % \SIrange soll die Einheit nur einmal anzeigen
    list-units  = repeat,       % \SIlist soll die Einheit wiederholen
}
%% bei siuntix funktioniert babel leider nicht
%% für englische Dokumente sollten diese Zeilen auskommentiert werden. 
\sisetup{
    range-phrase         = { bis },
    list-final-separator = { und },
%    list-pair-separator  = { und }, % an Uni noch nicht verfügbar
}

%% erlaubt es Bilddateien einzubinden
%% (ctable graphicx intern auch. Trotzdem ist es sinnvoll graphicx expilizt zu laden.
%%  Sonst entstehen schwehr verständliche Fehler, wenn ctable entfernt wird)
\usepackage{graphicx}
%% ermöglicht Bilder und Tabellen am eingegebenen Ort zu platzieren ([H])
\usepackage{float}
%% ermöglicht Unter-Bilder in einer figure-Umgebung
\usepackage{subfig}
%% Grafik-Dateien werden in den folgenden Ordnern gesucht
\graphicspath{{img/}}
%% Grafikdateien haben die folgenden Endungen (höchste Priorität zu erst)
\DeclareGraphicsExtensions{.pdf,.png,.jpg}

%% Vertikaler Abstand zwischen Absätzen, Beginn eines Absatzes nicht einrücken
\usepackage{parskip}
% \setlength{\parskip}{0.6em}   % Vertikaler Abstand zwischen Absätzen anpassen 
% \setlength{\parindent}{0em}   % Einrück-Abstand anpassen 

%% zeige Labels im Seitenrand. Dies ist praktisch um Verweise zu kontrollieren
\usepackage[final]{showkeys} % die Option 'final' deaktiviert die Ausgabe von showkeys

%% Seiten-Layout einstellen
\usepackage[
 a4paper,
 total={18cm,28cm},          % Breite und Höhe des Inhalt-Bereichs
 top=10mm, left=12mm,        % Ränder oben und links
 headsep=0mm,               % Abstand des unteren Rands der Kopfzeile vom oberen Rand des Inhalts
 footskip=10mm               % Abstand des unteren des Inhalts zum oberen Rand der Fusszeile
]{geometry}

%% Ermöglicht Links im PDF 
%%   sollte möglichst spät in der Präambel geladen werden
\usepackage[
 pdftex,                        % wir verwenden pdftex/pdflatex
 bookmarks=true,                % wir wollen auch im PDF-Reader ein Inhaltsverzeichnis
 bookmarksdepth=3,              % das Inhaltsverzeichnis soll 3 Tiefen enthalten
 colorlinks=true,               % Linktexte sollen Farbig sein 
 linkcolor=black,               % Links innerhalb des Dokuments bleiben schwarz
 citecolor=black,               % Links zu Quellenangaben bleiben ebenfalls schwarz
 urlcolor=blue,                 % URL-Linktexte sollen blau dargestellt werden
%  pdfborder={0 0 0}              % Links im PDF erhalten keinen Rahmen, nur nötig wenn colorlinks=false
]{hyperref}

%% Angaben für die PDF-Eigenschaften
\hypersetup{
  pdfauthor = {Pascal Horat, Steve Gerome Kamga, G"okhan Kaya},
  pdftitle = {},
  pdfsubject = {},
  pdfkeywords = {}
}


%% definiert \cref: Referenzen mit korrekter Bezeichnung (z.B. "Abbildung 1")
%%   die Nummer alleine ist weiter mittels \ref verfügbar
%% muss NACH 'hyperref' geladen werden
\usepackage[german]{cleveref}
% \usepackage[english, capitalise]{cleveref}

\usepackage{tabulary}
\usepackage{array}

%% Angaben für \maketitle
\title{Regeln TKI}
\author{Pascal Horat, Steve Gerome Kamga, Gökhan Kaya}
% \date{7. Mai 2013}             % ohne Angabe wird das heutig Datum verwendet

\begin{document}

\maketitle

Das Ziel dieses Dokuments ist das Festlegen allgemeiner Regeln, die für die zuk"unftige Teamarbeit als verbindlich gelten.\\

\begin{center}
\includegraphics[scale=0.2]{clock}\\
\end{center}


\begin{itemize}
\item P"unktlichkeit ist wichtig. Bei mehr als 10 Minuten Versp"atung, ist dies den anderen per WhatsApp zu melden
\end{itemize}

\begin{itemize}
\item Falls ein Auftrag ohne akzeptable Begr"undung nicht erledigt wird, muss dieser auf die n"achste Woche nachgeholt werden. Falls er auch dann nicht erledigt wurde, wird eine Konsequenz mit dem Dozenten erarbeitet
\end{itemize}

\begin{itemize}
\item Wenn nicht anders abgemacht, treffen wir uns jeden Montag in der Vorlesung. Falls weitere Termine notwendig sind, werden zus"atzliche Sitzungen einberufen \\\\
\end{itemize}

\begin{itemize}
\item Als Kommunikationskan"ale dienen unsere Whatsapp-Gruppe und das schulinterne E-Mail. Als Datenablage und Versionskontrolle verwenden wir GitHub. Bei gr"osseren Files wird die eingerichtete OneDrive-Cloud hinzugezogen\\\\
\end{itemize}

\begin{itemize}
\item Um Dokumente zu setzen, wird \LaTeX verwendet. Die n"otigen Vorlagen werden, falls als sinnvoll erachtet, erstellt\\\\
\end{itemize}

\begin{itemize}
\item Da unsere Gruppenstruktur keinen eigentlichen Chef aufweist, werden folgende Verantwortungsbereiche definiert ("Anderungen bei Einstimmigkeit vorbehalten):\\


\begin{itemize}
\item Projektkoordinator: Pascal Horat
\item LaTeX-Vorlagen: Pascal Horat
\item MS-Project: Steve Gerome Kamga
\item Git/Github: G"okhan Kaya  
\item Projektbericht: Pascal Horat
\item Kostenveranschlagung: G"okhan Kaya
\item Vortragsplanung: Steve Gerome Kamga
\item Teamreview: alternierend
\item Sitzungschef: alternierend
\end{itemize}
\end{itemize}

\begin{itemize}
\item Jede Meinungsverschiedenheit wird besprochen und falls keine Einigung erzielt wird, nach relativer Mehrheitswahl entschieden\\\\
\end{itemize}

\begin{itemize}
\item Unsere Teamphilosophie ist es, mit einem gegebenen Zeitaufwand ein m"oglichst gutes Produkt abzuliefern. Das Ziel ist es, m"oglichst viele Arbeiten w"ahrend der Vorlesungszeit erledigen zu können\\\\
\end{itemize}

\begin{itemize}
\item Falls eine Person das angestrebte Ziel vernachl"assigt, wird eine zweiw"ochige Frist angesetzt. Hat sich in dieser Frist die Qualit"at der Produkte nicht verbessert, werden weitere Schritte in Absprache mit dem Dozenten einigeleitet\\\\
\end{itemize}
 
\end{document}
