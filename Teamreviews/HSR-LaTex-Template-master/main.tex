%
% HSR LaTex Template
% Copyright 2012, Florian Bentele
%
% Complete LaTex template for thesis at HSR, customized
% for Prof. Dr. Peter Heinzmann
%
%
% This document is free software: you can redistribute
% it and/or modify it under the terms of the GNU
% General Public License as published by the Free
% Software Foundation, either version 3 of the License,
% or (at your option) any later version.
%
% This document is distributed in the hope that it will
% be useful, but WITHOUT ANY WARRANTY; without even the
% implied warranty of MERCHANTABILITY or FITNESS FOR A
% PARTICULAR PURPOSE. See the GNU General Public
% License for more details.
%
% You should have received a copy of the GNU General
% Public License along with this document. If not, see
% <http://www.gnu.org/licenses/>.
%

\documentclass[11pt,twoside]{hsrthesis}
\usepackage{times}

\usepackage{array}
\usepackage{colortbl}
\usepackage{graphicx}
\usepackage{tikz}

\colorlet{helpful}{lime!70}
\colorlet{harmful}{red!30}
\colorlet{internal}{yellow!20}
\colorlet{external}{cyan!30}
\colorlet{S}{helpful!50!internal}
\colorlet{W}{harmful!50!internal}
\colorlet{O}{helpful!50!external}
\colorlet{T}{harmful!50!external}

\newcommand{\texta}{Helpful\par \tiny (to achieve the objective)}
\newcommand{\textb}{Harmful\par \tiny (to achieve the objective)}
\newcommand{\textcn}{\rotatebox[origin=c]{90}{\parbox[t]{3cm}{\centering Internal origin\\ \tiny (product\slash company attributes)\par}}}
\newcommand{\textdn}{\rotatebox[origin=c]{90}{\parbox[b]{3cm}{\centering External origin\\ \tiny (environment\slash market attributes)\par}}}

\newcommand{\texts}{\makebox[0pt][c]{\parbox[t]{0.2\textwidth}{\centering strength 1\par strength 2}}}
\newcommand{\textw}{\makebox[0pt][c]{\parbox[t]{0.2\textwidth}{\centering weakness 1\par weakness 2}}}
\newcommand{\texto}{\makebox[0pt][c]{\parbox[t]{0.2\textwidth}{\centering opportunity 1\par opportunity 2}}}
\newcommand{\textt}{\makebox[0pt][c]{\parbox[t]{0.2\textwidth}{\centering threat 1\par threat 2}}}


\newcommand{\back}[1]{\tikz\node[overlay,text=#1!60!black,font=\fontsize{60}{70}\selectfont](char) at (0,-0.1) {#1};}
\newcommand\mycolor[1]{\cellcolor{#1}}
\newcolumntype{C}[1]{>{\centering\arraybackslash}m{#1}}
\makeindex

% do this here, so you can \gls{...} to it
%%add new glossaryentries here...

\newglossaryentry{sqlite}{
	name=SQLite,
	description={SQLite ist eine Datenbankengine, welche ohne Konfiguration auskommt. Es handelt sich dabei um eine Datenbank in einer Datei},
	first={SQLite}
}
%\makeglossaries
\usepackage{rotating}
\begin{document}

\newcommand{\thesistitle}{Teamreview 2}
\newcommand{\thesisauthora}{Pascal Horat, Steve Gerome Kamga, Gökhan Kaya}
\newcommand{\thesisauthorb}{}
\newcommand{\thesisauthorc}{}
\newcommand{\professor}{Prof. Dr. Some Body}
\newcommand{\thesistype}{Bachelorarbeit}
\newcommand{\departement}{Abteilung Informatik}
\newcommand{\school}{Hochschule für Technik Rapperswil}
\newcommand{\term}{Frühjahrssemester 2017}
\newcommand{\thedate}{21. März 2017}
\newcommand{\timeperiode}{21.02.2012 - 01.06.2012}
\newcommand{\partner}{-}
\newcommand{\workload}{-}
\newcommand{\linktothesis}{https://moodle.hsr.ch}

\setlength{\oddsidemargin}{20mm}
\maketitle
\setlength{\oddsidemargin}{20mm}


\tableofcontents


% The main content
%%%%%%%%%%%%%%%%%%
%\begin{comment}
\chapter*{Examples}

\section*{Glossary Example}

\gls{sqlite} ist ein Glossar Eintrag. Lorem ipsum dolor sit amet, consetetur sadipscing elitr, sed diam nonumy eirmod tempor invidunt ut labore et dolore magna aliquyam erat, sed diam voluptua. At vero eos et accusam et justo duo dolores et ea rebum.

\section*{Bibliography and Citation Example}

Dies ist ein Zitat aus einem Buch\cite{Matthews201111}. Lorem ipsum dolor sit amet, consetetur sadipscing elitr, sed diam nonumy eirmod tempor invidunt ut labore et dolore magna aliquyam erat, sed diam voluptua. At vero eos et accusam et justo duo dolores et ea rebum.

Lorem ipsum dolor sit amet, consetetur sadipscing elitr, sed diam nonumy eirmod tempor invidunt ut labore et dolore magna aliquyam erat, sed diam voluptua. At vero eos et accusam et justo duo dolores et ea rebum. Stet clita kasd gubergren, no sea takimata sanctus est Lorem ipsum dolor sit amet. Lorem ipsum dolor sit amet, consetetur sadipscing elitr, sed diam nonumy eirmod tempor invidunt ut labore et dolore magna aliquyam erat, sed diam voluptua. At vero eos et accusam et justo duo dolores et ea rebum. Stet clita kasd gubergren, no sea takimata sanctus est Lorem ipsum dolor sit amet. Lorem ipsum dolor sit amet, consetetur sadipscing elitr, sed diam nonumy eirmod tempor invidunt ut labore et dolore magna aliquyam erat, sed diam voluptua. At vero eos et accusam et justo duo dolores et ea rebum. Stet clita kasd gubergren, no sea takimata sanctus est Lorem ipsum dolor sit amet. 

Duis autem vel eum iriure dolor in hendrerit in vulputate velit esse molestie consequat, vel illum dolore eu feugiat nulla facilisis at vero eros et accumsan et iusto odio dignissim qui blandit praesent luptatum zzril delenit augue duis dolore te feugait nulla facilisi. Lorem ipsum dolor sit amet, consectetuer adipiscing elit, sed diam nonummy nibh euismod tincidunt ut laoreet dolore magna aliquam erat volutpat. 

Ut wisi enim ad minim veniam, quis nostrud exerci tation ullamcorper suscipit lobortis nisl ut aliquip ex ea commodo consequat. Duis autem vel eum iriure dolor in hendrerit in vulputate velit esse molestie consequat, vel illum dolore eu feugiat nulla facilisis at vero eros et accumsan et iusto odio dignissim qui blandit praesent luptatum zzril delenit augue duis dolore te feugait nulla facilisi. 


\section*{Table Examples}
\subsection*{Automagical Column Widths}
\begin{tabularx}{\textwidth}{X X X X} \beforeheading
\heading{Heading 1} & \heading{Heading 2} & \heading{Heading 3} & \heading{Heading 4} \\\afterheading
Cell 1,1 & Cell 1,2 & Cell 1,3 & Cell 1,4 \\\normalline
Cell 2,1 & Cell 2,2 & Cell 2,3 Vel illum dolore eu feugiat nulla facilisis at vero eros et accumsan et iusto odio dignissim qui blandit praesent luptatum zzril delenit augue duis dolore te feugait nulla facilisi. & Cell 2,4 \\\normalline
Cell 3,1 Duis autem vel eum iriure dolor in hendrerit in vulputate velit esse molestie consequat. & Cell 3,2 & Cell 3,3 & Cell 3,4 \\\lastline
\end{tabularx}

\subsection*{Column Alignment \& Filler}
\begin{tabularx}{\textwidth}{l c r X} \beforeheading
\heading{Left Aligned} & \heading{Centered} & \heading{Right Aligned} & \heading{Filler} \\\afterheading
Cell 1,1 & Cell 1,2 & Cell 1,3 & Cell 1,4 Ut wisi enim ad minim veniam, quis nostrud exerci tation ullamcorper  \\\normalline
Cell 2,1 & Cell 2,2 & Cell 2,3 & Cell 2,4 Ut wisi enim ad minim veniam, quis nostrud exerci tation ullamcorper  \\\normalline
Cell 3,1 & Cell 3,2 & Cell 3,3 & Cell 3,4 Ut wisi enim ad minim veniam, quis nostrud exerci tation ullamcorper  \\\lastline
\end{tabularx}
\end{comment}
%03introduction.tex

\chapter{Ziel dieses Dokumentes}

Das Ziel dieses Dokuments ist es, die im HSR-Modul Teamkommunikation für Ingenieure erlernten Teamrollen-Modelle anwenden zu können und diese den Teammitgliedern zuzuordnen. So sollen potenzielle Stärken und Schwächen des Teams entdeckt und Konsequenzen eingeleitet werden. 

% \input{chapter/04mychapter}
% ...

\chapter{Selbsteinschätzung gemäss Belbin $^{[2]}$}
\section{Auftrag}

Mittels einer SWOT-Analyse$^{[1]}$ soll beschrieben werden:
\begin{itemize}
\item Welche Interessen und Ziele verfolge ich bei der Projektarbeit?
\item Wie schätze ich meine eigene Position im Team ein?
\end{itemize}


\chapter{Selbsteinschätzung gemäss Belbin}
\begin{tabular}{c*{2}{C{0.2\textwidth}}}
                            &\cellcolor{helpful} \texta  & \cellcolor{harmful} \textb \\
\cellcolor{internal}\textcn & \mycolor{S}\back{S} \texts & \mycolor{W}\back{W} \textw \\
\cellcolor{external}\textdn & \mycolor{O}\back{O} \texto & \mycolor{T}\back{T} \textt
\end{tabular}
\\
\newline
Der Übersichtlichkeit halber werden die einzelnen Punkte nicht in die obige Tabelle geschrieben, sondern anschliessend aufgelistet.

\section{Vergleich Selbst- und Fremdeinschätzung}
\begin{itemize}
\item Mein Ziel ist es später in einer Firma Führungstätigkeiten zu übernehmen. Darum ist es besonders interessant für mich, zu lernen, wie effektive Teams zusammengestellt werden können
\item Ich arbeite strukturiert. Für das Bearbeiten des Projektauftrags im Team ist das ein Vorteil
\item Schon mehrere Male habe ich in Gruppen Produkte erarbeitet, oft habe ich dabei eine führende Rolle eingenommen
\item Unser Team besteht aus Elektrotechnik- und Maschinenbaustudenten. Die jeweiligen Ausbildungsinhalte und das Denken sind sehr ähnlich 
\item Das Verstehen von schwammig definierten Aufträgen bereitet mir weniger Schwierigkeiten als anderen
\item Durch meine Führungserfahrung kann ich gut als Chef oder Koordinator$^{[2]}$ eingesetzt werden
\end{itemize}

\subsection{weakness}
\begin{itemize}
\item Manchmal bereitet es mir Schwierigkeiten auf Vorschläge anderer eingehen zu können, weil ich meine schon als die beste Variante erachte
\item Ich versuche automatisch die Rolle des Chefs einzunehmen, obwohl das noch gar nicht besprochen wurde$^{[3]}$ 
\item Den anderen Teammitgliedern höre ich zeitweise nicht sehr aktiv zu
\item Das Arbeiten fällt mir am leichtesten in einem stillen Raum, durch andere Teams bin ich schnell abgelenkt
\item Ich fälle Entscheidungen für das Team alleine, ohne die anderen Mitglieder in den Entscheidungsfindungsprozess miteinzubeziehen 
\end{itemize}

\subsection{opportunity}
\begin{itemize}
\item Da wir nicht das einzige Team sind welches diesen Auftrag bearbeitet, kann ein Austausch von grossem Nutzen sein
\item Der Dozent hat in diesem Bereich schon mehrjährige Arbeitserfahrung. Er kann kompetent Auskunft geben
\item Sobald wir ein einigermassen eingespieltes Team sind, kann das Erledigen von Teamarbeiten grossen Spass bereiten, der Lernerfolg würde dadurch umso grösser
\end{itemize}


\subsection{threat}
\begin{itemize}
\item Die Arbeitslast wird aufgrund von Aufträgen in anderen Fächern zu gross, die gefragten Produkte werden von uns nicht mehr, oder nur ungenügend erstellt
\item Die Motivation eines einzelnen oder der Gruppe lässt aus externen, nicht beeinflussbaren Gründen nach (z.B. Freundin, Familie usw.)
\item Meine Absenz auf Grund von Militärdienst (dreieinhalb Wochen, am Ende des Semesters) hat grossen negativen Einfluss auf den Projektbericht und andere Aufträge im Modul
\end{itemize}

\chapter{Lernbilanz 2: Ich und die Konflikte im Team}

\chapter{Lernbilanz 3: Mich und das Team entwickeln} 

\chapter{Referenzen}

\begin{itemize}
\item[[1]] Folie PM	2.1.14, S. 46 in TKI{\_}2017.02.20.2.pdf
\item[[2]] Folie TE 2.2.2.2, S. 104 in TKI{\_}2017.02.27.pdf
\item[[3]] siehe Storming, Folie TE 1.1.6, S. 27 in TKI{\_}2017.02.20.1.pdf
\end{itemize}
 
% List of figures & glossary
%%%%%%%%%%%%%%%%%%%%%%%%%%%%
%\listoffigures
%\printglossary[style=altlist,title=Glossar]

% Bibliography
%%%%%%%%%%%%%%
\bibliographystyle {alpha}
\bibliography{index/bibliography}


% Attached sources
%%%%%%%%%%%%%%%%%%
 \input{attachments/attachments}


\end{document}