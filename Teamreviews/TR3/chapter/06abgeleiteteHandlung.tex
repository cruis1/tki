%06chapterVergleich.tex

\chapter{Abgeleitete Handlungen}

Nach den obigen Analysen und Diskussionen hat sich unser Team, um die zukünftige Effektivität zu verbessern, auf folgende vier Punkte geeinigt.  

\subsection*{Frühzeitiges Informieren über die Aufträge}

Um zukünftig mehrheitlich agieren zu können und nicht reagieren zu müssen, ist es für unser Team unabdingbar, die Auftragslage so früh und detailliert wie möglich zu klären. Dies beinhaltet unter anderem, dass wir uns über die Aufträge, welche für die übernächste Woche anstehen Infomieren, so dass wir in der letzten Vorlesung vor Abgabe dem Dozenten entsprechende Fragen gestellt werden könnten. 

\subsection*{Mehr sachliche Kritik}  

Nur wenn Regelmössig sachliche Kritik angebracht wird, kann sich unser Team laufend verbessern. Diese sollte als Möglichkeit sich selber zu verbessern zu können angeschaut werden und nicht als persönlicher Angriff aufgefasst werden. 

\subsection*{Bessere Koordination und Aufgabenteilung}

Da das AC-Projekt recht umfangreich sein wird, muss die Aufgabenteilung besser koordiniert werden. Es ist überhaupt nicht effizient alle Arbeiten als Dreierteam zu erledigen. 

\subsection*{Öfters gegenseitiges Nachfragen bei Unklarheiten}

In einem Projekt sollten nach einer gewissen Zeit alle Teammitglieder einen abgeglichenen Wissensstand haben. Aus diesem Grund muss bei Unklarheiten, seien diese administrativer, technischer oder inhaltlicher Natur, sofort bei Teammitgliedern nachgefragt werden.


\nocite{lencioni2010five}