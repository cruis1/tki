%06chapterVergleich.tex

\chapter{Abgeleitete Handlungen}

Nach den obigen Analysen und Diskussionen hat sich unser Team, um die zukünftige Effektivität zu verbessern, auf folgende vier Punkte geeinigt.  

\subsection*{Frühzeitiges Informieren über die Aufträge}

Um zukünftig mehrheitlich agieren zu können und nicht reagieren zu müssen, ist es für unser Team unabdingbar, die Auftragslage so früh und detailliert wie möglich zu klären. Dies beinhaltet unter anderem, dass wir uns über die Aufträge, welche für die übernächste Woche anstehen Infomieren, so dass wir in der letzten Vorlesung vor Abgabe dem Dozenten entsprechende Fragen gestellt werden könnten. 

\subsection*{Mehr sachliche Kritik}  

Nur wenn Regelmössig sachliche Kritik angebracht wird, kann sich unser Team laufend verbessern. Diese sollte als Möglichkeit sich selber zu verbessern zu können angeschaut werden und nicht als persönlicher Angriff aufgefasst werden. 

\subsection*{Bessere Koordination und Aufgabenteilung}

Da das AC-Projekt recht umfangreich sein wird, muss die Aufgabenteilung besser koordiniert werden. Es ist überhaupt nicht effizient alle Arbeiten als Dreierteam zu erledigen. 

\subsection*{Öfters gegenseitiges Nachfragen bei Unklarheiten}

In einem Projekt sollten nach einer gewissen Zeit alle Teammitglieder einen abgeglichenen Wissensstand haben. Aus diesem Grund muss bei Unklarheiten, seien diese administrativer, technischer oder inhaltlicher Natur, sofort bei Teammitgliedern nachgefragt werden.


\begin{comment}
Das Ziel in diesem Kapitel ist es, die in den vorhergehenden Kapiteln ausgearbeiteten Selbst- und Fremdeinschätzungen in Kontrast zu setzen. Dazu wird die jeweilige Selbsteinschätzung eines jeden Teammitgliedes mit den Fremdeinschätzungen der beiden anderen verglichen. Bei grossen Abweichungen werden diese versucht zu Begründen.

\subsection*{Pascal Horat}
Bei Pascal hat es von allen Vergleichen die geringsten Unterschiede gegeben. So hat er sich selbst als Koordinator eingeschätzt, diese Rolle wurde auch von den anderen beiden Teammitgliedern für ihn als erstes genannt. Siehe Kapitel \ref{Fremdeinschaetzung} ab Seite \pageref{Fremdeinschaetzung}. 

\subsection*{Steve Gerome Kamga}
Gerome wurde von Gökhan als Vollender eingeschätzt, wobei er dies auch als dritter Punkt in seiner Selbsteinschätzung erwähnte. \\
Von Pascal wurden ihm die Rollen des Pragmatikers und Ausgleichers zugeteilt. Als Ausgleicher konnte sich Gerome jedoch nicht bewerten, so war diese Rolle nur an vierter Stelle in der Auswertung.

\subsection*{Gökhan Kaya}
Die Teamkollegen von Gökhan haben ihn als Ausgleicher und Bewerter betrachtet. Bei der Rolle des Ausgleichers hatte es eine grosse Übereinstimmung gegeben, jedoch hatte Gökhan bei seiner Selbsteinschätzung den Bewerter an letzter Stelle. Die Rolle des Bewerters wurde Gökhan von Gerome anhand des Belbin Fragebogens zugeteilt. Gemäss Rollenbeschreibung von Belbin zeichnet den Bewerter seine zurückhaltende Position, aber auch sein klares Urteil bezüglich Teamentscheidungen aus. Dieser Auszug stimmt laut Aussage des Beurteilenden sehr genau. Hingegen wird Gökhan von ihm überhaupt nicht als Bremser, wie dies in der Rollenbeschreibung steht, eingeschätzt.
\end{comment}