%04chapterFremdeinsch.tex

\chapter{Fremdeinschätzung}
Nach unseren individuell erstellten Selbsteinsch"atzungen, folgen nun die Fremdeinsch"atzungen, welche f"ur
den Vergleich von Selbst- und Fremdbild unabdingbar sind.
Dazu hat jedes Teammitglied seine eigene Form der Bearbeitung gew"ahlt. Dies war einerseits der Selbsteinsch"atzungstest andererseits wurden anhand von spezifisch beobachteten Situationen eine Teamrollenzuteilung
vorgenommen.
\subsection*{Pascal Horat}
\subsection*{Steve Gerome Kamga}

\subsection*{Gökhan Kaya}

Ich habe meine zwei Teamkollegen folgendermassen eingeschätzt:\\

\textbf{Pascal} war für mich der Koordinator und zwar aus den folgenden erlebten Gründen:
\begin{enumerate} 
\item{Meist sagt Pascal am Anfang der Sitzung was zu tun ist und gibt uns einen kleinen Überblick. Er ist somit sehr gut organisiert.}
\item{Er ist meistens der erste, der sich mit dem Dozenten in Verbindung setzt, wenn etwas noch unklar ist. Die nötigen Informationen leitet er uns anschliessend weiter.}
\item{Bereits bei der ersten Vorlesung ist mir aufgefallen, dass Pascal sich sehr vieles notiert und teilweise nach der Vorlesung ebenfalls mitteilt, was er sich notiert hat.}
\end{enumerate}

\textbf{Gerome} war für mich ganz klar der Vollender. Dies aus den folgenden drei Gründen:
\begin{enumerate} 
\item{Gleich nach der Teambildung konnte Gerome noch nicht auf moodle zugreifen. Er hat sich nach einem Tag bereits sofort gemeldet und mitgeteilt, dass bei ihm nun alles soweit funktioniert. Dies obwohl es nicht nötig gewesen wäre.}
\item{Ebenfalls hat nach der ersten Aufgabenteilung sofort auf Whatsapp geschrieben und uns mitgeteilt, dass er sein Auftrag erledigt hat. Pascal und ich empfanden dies hingegen nicht nötig.}
\item{Als wir am letzten Tag vor der Abgabe noch bis spät die Lernbilanzen fertig gestellt haben, hat Gerome die ersten zwei Stunden ausschliesslich recherchiert, um das bestmögliche Produkt abzugeben. Dies obwohl wir nur wenig zeit hatten. Dabei machte es ihm auch nichts aus, bis spät in die Nacht hinein zu arbeiten.}
\end{enumerate}
