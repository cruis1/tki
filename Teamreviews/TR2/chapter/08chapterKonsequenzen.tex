%08chapterKonsequenzen.tex

\chapter{Konsequenzen}

Hier geht es nun darum, die im Kapitel Zusammenfassung notierten Inhalte in effektive Konsequenzen umzuwandeln.

Da unser Team in der aktuellen Besetzung eine vielfältige Rollenverteilung aufweist, gestaltet es sich schwierig direkte Konsequenzen, die zu einer verbesserten Teameffektivität führen, abzuleiten. Würde sich die Auftragslage in nächster Zeit drastisch ändern, könnten wir aber zu spüren bekommen, dass in unserem Team die innovative Kraft fehlt. Würde dieser Fall eintreten, müssten wir wahrscheinlich mit dem Dozenten Kontakt aufnehmen, um an neue Ideen zu kommen. 

Obwohl sich durch die Analyse der Rollenverteilung nicht wirklich direkt umsetzbare Konsequenzen herauskristallisiert haben, sind wir in der dadurch angeregten Diskussion doch auf einige Dinge gestossen, die für eine bessere Effektivität des Teams sorgen könnten.
So ist zum Beispiel zum Vorschein gekommen, dass Pascal vermehrt auf Vorschläge der anderen hören und nicht seine Version schon im Voraus als beste einstufen sollte. Auch wurde die Pünktlichkeit angesprochen, welche ein wichtiger Bestandteil unseres Regeldokumentes ist. Als letztes haben wir uns vorgenommen Aufträge früher zu analysieren und etwaige Friktionen frühzeitig mit dem Dozenten zu besprechen.