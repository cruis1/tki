%04chapterSelbsteinsch.tex

\chapter{Selbsteinschätzung}

In diesem Kapitel wird beschrieben, welche Rollen die Teammitglieder mittels Selbsteinsch"atzung erhalten. Dazu wird ein Test verwendet, welcher urspr"unglich von R.M. Belbin \cite{belbin1981management} entwickelt und an der Universit"at Regensburg ins Deutsche %\cite{3} übersetzt wurde. Dieser Test 

\paragraph{Gökhan}

Meine Auswertung der Selbsteinschätzung gemäss Belbin ergab die folgende Punkteverteilung:
\begin{tabular}{ c c c }
  Visionär & Akquisiteur & Prozessgestalter & Koordinator & Bewerter & Ausgleicher & Pragmatiker & Vollender & SPezialist \\
  2 & 5 & 8 & 10 & 2 & 19 & 7 & 6 & 11 \\
\end{tabular}

Somit zu sehen, dass folgende drei Teamrollen herausstechen: \\
\renewcommand{\labelenumi}{\roman{enumi}} 
\begin{enumerate} 
\item Ausgleicher 
\item Spezialist
\item Koordinator
\end{enumerate}

Wobei besondern die Rolle des Ausgleichers mit 19 Punkten hervorsticht. Dies war für mich keine Überraschung, da ich während den Teamsitzungen bereits gemerkt habe, dass ich mich sehr für den Teamzusammenhalt einsetze. Dies ist der Fall, da für mich eine gute Atmosphäre das wichtigste bei einer Teamarbeit ist.


\begin{comment}
\section{Lorem}
Lorem ipsum dolor sit amet, consectetur adipiscing elit. Sed gravida mollis placerat. Sed congue iaculis massa vitae dapibus. Fusce sed felis lorem. Suspendisse purus diam, sollicitudin vitae imperdiet ac, placerat eu metus. In luctus, metus vel dictum hendrerit, diam lacus cursus enim, eu porta augue lacus non metus. Pellentesque habitant morbi tristique senectus et netus et malesuada fames ac turpis egestas. Nullam nec orci eget metus pulvinar sagittis. Vestibulum ante ipsum primis in faucibus orci luctus et ultrices posuere cubilia Curae; Sed turpis lorem, aliquet eu ornare non, viverra ac urna.

Praesent libero lectus, ultrices eget pharetra sed, sollicitudin et est. Pellentesque quis urna eget lorem sodales venenatis eget nec quam. In sagittis aliquam auctor. Phasellus vitae ipsum purus, sit amet imperdiet nunc. Pellentesque habitant morbi tristique senectus et netus et malesuada fames ac turpis egestas. Ut malesuada nibh ut lectus scelerisque sed iaculis lectus varius. Nulla blandit turpis tortor. Nulla facilisi. Cum sociis natoque penatibus et magnis dis parturient montes, nascetur ridiculus mus. Nam leo ante, porta vel scelerisque at, volutpat eu sapien. Aliquam viverra adipiscing sapien et porta. Sed quis diam ut sem tincidunt consectetur varius non dolor. Fusce fermentum, quam vitae suscipit euismod, leo erat malesuada ante, ac consequat est lacus eget enim. Proin lacinia justo et est vehicula adipiscing rhoncus lacus mollis.	
\end{comment}